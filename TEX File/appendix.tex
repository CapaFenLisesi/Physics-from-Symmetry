%!TEX root = main.tex
%!TEX encoding = UTF-8 Unicode

\appendix
\part{Appendix 附录}
\chapter[矢量代数]{Vector calculus 矢量代数}

It is often useful in physics to describe the position of some object
using three numbers

\section[基矢]{Basis Vectors 基矢}

\section[坐标系变换]{Change of Coordinate Systems 坐标系变换}

\section[矩阵乘法]{Matrix Multiplication 矩阵乘法}

\section[标量]{Scalars 标量}

\section[左手/右手坐标系]{Right-handed and Left-handed Coordinate Systems 左手/右手坐标系}

\chapter[微积分]{Calulus 微积分}

\section[Product Rule]{Product Rule 莱布尼兹律}

\section[分部积分]{Integration by Parts 分部积分}

\section[Taylor 级数]{The Taylor Series\quad Taylor 级数}

\section[级数]{Series 级数}
\subsection[几个重要的级数]{Important Series 几个重要的级数}
\subsection[分裂求和]{Splitting Sums 分裂求和}
\subsection[Einstein 求和约定]{Einstein’s Sum Convention\quad Einstein 求和约定}

\section[指标记号]{Index Notation 指标记号}
\subsection[哑指标]{Dummy Indices 哑指标}
\subsection[带多个指标的对象]{Objects with more than One Index 带多个指标的对象}
\subsection[对称/反对称指标]{Symmetric and Antisymmetric Indices 对称/反对称指标}
\subsection[对称和反对称求和]{Antisymmetric $\times$ Symmetric Sums 对称和反对称求和}
\subsection[两个重要的符号]{Two Important Symbols 两个重要的符号}

\chapter[线性代数]{Linear Algebra 线性代数}
\section[基本的变换]{Basic Transformations 基本的变换}
\section[矩阵指数函数]{Matrix Exponential Function 矩阵指数函数}
\section[行列式]{Determinants 行列式}
\section[本征值与本征向量]{Eigenvalues and Eigenvectors 本征值与本征向量}
\section[对角化]{Diagonalization 对角化}

\chapter[其他数学概念]{Additional Mathematical
Notions 其他数学概念}
\section[Fourier 变换]{Fourier Transform\quad Fourier 变换}
\section[Delta 分布]{Delta Distribution\quad Delta 分布}
