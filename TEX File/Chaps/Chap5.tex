%!TEX encoding = UTF-8 Unicode

%----------------------------------------------------------------------------------------
%	CHAPTER 1
%----------------------------------------------------------------------------------------

\chapterimage{chapter_head_1.pdf} % Chapter heading image

\chapter[测量]{Measuring Nature 测量}\label{chap5}

我们现在已经发现了对称性与守恒量之间的关系了,那么我们接下来就可以利用这种联系。用更专业的话说,Noether定理建立了对称变换的生成元与一个守恒量之间的联系。我们本章将利用这种联系。

守恒量常常被物理学家们用来描述物理系统,因为无论这个系统经历了怎样复杂的变化,守恒量都是一样的。比如,物理学家们为了描述一个实验会使用动量,能量或角动量。Noether定理提示我们了一个方向和一个至关重要的想法:我们在挑选哪些量我们用来描述自然的时候,我们实际上在看伴随着其对应的生成元:

\begin{align}
\text{物理量}\Rightarrow\text{对应对称性的生成元}
\end{align}

我们会看到,这种选择会自然地引导我们给出给出量子理论。

\section[量子力学中的算符]{Operators of Quantum Mechanics 量子力学中的算符}\label{sec5.1}

{\it 惯例上用一个尖帽子$\hat{O}$来表示一个算符}

拉格朗日量在空间平移操作的生成元的作用下不变导致我们得出动量守恒。因此,我们可以得到
\[\text{动量}\hat{p}_i\to\text{空间平移操作的生成元} - i\partial_i \]





