%!TEX encoding = UTF-8 Unicode

%----------------------------------------------------------------------------------------
%	CHAPTER 1
%----------------------------------------------------------------------------------------

\chapterimage{chapter_head_1.pdf} % Chapter heading image

\chapter{Introduction 简介}\label{chap1}

\section{What we Cannot Derive 得不到的事情}

在我们开始讲我们能从对称性里面了解到什么之前,我们首先澄清一下我们需要在我们的理论中人为的加一些什么东西。首先,目前没有任何理论可以得到自然界的常数。这些常数需要从实验中提取出来,比如各种相互作用的耦合常数啊,基本粒子的质量啊这种的。

除了这些,我们还有一些东西解释不了:{\bf 数字$3$}。这不是术数的那种神秘主义的东西,而是我们不能解释所有的直接与数字$3$相联系的限制。比如:

\begin{itemize}
\item 对应三种标准模型描述的基本作用力有三种规范理论\mpar{如果你不理解这个简介中的某些名词,比如规范理论或者二重覆盖,不需要太过担心。本书将会详尽的解释,在这里提到只是为了完整性。}。这些力是由分别对应于对称群$U(1), SU(2)$和$SU(3)$的规范理论描述的。为什么没有对应$SU(4)$带来的基本作用力?没人知道!
\item 轻子有三代,夸克也有三代。为什么没有第四代?我们只能从实验\mpar{比如,现在宇宙中元素的丰度是依赖于代的数量的。更进一步,对撞机实验中有对此的很强的证据。(见Phys. Rev. Lett. 109, 241802)}中知道没有第四代。
\item 我们只在拉格朗日量里面包含$\Phi$的最低三阶$(\Phi^0, \Phi^1, \Phi^2)$,其中$\Phi$指代一些描述我们的物理系统的东西,是个通称,而这个拉格朗日量则是被我们用来得到我们的描述自由(=无相互作用)场/粒子的靠谱的理论的。
\item 我们只用三个基本的Poincare群双覆盖的表示,分别对应自旋$0, \tfrac{1}{2}$和$1$。没有基本粒子的自旋是$\tfrac{3}{2}$。
\end{itemize}

在现代的理论中,这些是我们必须手动增加的假定。我们从实验上知道这些假定是正确的,但是目前为止我们没有更深刻的原理告诉我们为什么我们需要到$3$就停。

除此之外,还有两件事情我们没法从对称性中得到,但是他们对于一个严谨的理论来说有时必须被考虑到的:

\begin{itemize}
\item 我们只允许在拉格朗日量中引入尽可能的最低阶的非平凡的微分算符$\partial_\mu$
\end{itemize}