%!TEX encoding = UTF-8 Unicode

%----------------------------------------------------------------------------------------
%	CHAPTER 1
%   translator: laserdog
%   proofreader: SI
%----------------------------------------------------------------------------------------

\chapterimage{chapter_head_1.pdf} % Chapter heading image

\begin{quote}%quote for every part.I have trouble inserting it into the part cover,so it's now there as a workaround.If you have a better way,please tell me.
{\bf “真相总比你想的简单”}\marginpar{"The truth always turns out to be simpler than you thought."}


\begin{flushright}
-- Richard P. Feynman\\
as quoted by\\
K. C. Cole. {\it Sympathetic Vibrations.}\marginpar{%
K. C. Cole.\\ {\it Sympathetic Vibrations.}
Bantam,reprint edition,\\10 1985.\\
ISBN 9780553342345%
}
\end{flushright}
\end{quote}

\chapter[简介]{Introduction 简介}\label{chap1}

\section[推不出来的事情]{What we Cannot Derive 推不出来的事情}\label{sec1.1}

在开始讲我们能从对称性里面了解到什么之前,我们首先说明需要在理论中人为加一些什么东西。首先,目前没有任何理论可以得到自然界的常数。这些常数需要实验测定,比如各种相互作用的耦合常数啊,基本粒子的质量啊这种的。

除了这些,我们还有一些东西解释不了:{\bf 数字$3$}。这不是术数的那种神秘主义的东西,而是我们不能解释所有的直接与数字$3$相联系的限制。比如:

\begin{itemize}
\item 对应三种标准模型描述的基本作用力有三种规范理论\mpar{如果你不理解这个简介中的某些名词,比如规范理论或者双覆盖,不需要太过担心。本书将会详尽的解释,在这里提到只是为了完整。}。这些力是由分别对应于对称群$\mathcal{U}(1), \mathcal{SU}(2)$和$\mathcal{SU}(3)$的规范理论描述的。为什么没有对应$\mathcal{SU}(4)$带来的基本作用力?没人知道!
\item 轻子有三代,夸克也有三代。为什么没有第四代?我们只能从实验\mpar{比如,现在宇宙中元素的丰度是依赖于代的数量的。更进一步,对撞机实验中有对此的很强的证据。(见Phys. Rev. Lett. 109, 241802)}中知道没有第四代。
\item 我们只在拉格朗日量里面包含$\Phi$的最低三阶$(\Phi^0, \Phi^1, \Phi^2)$,其中$\Phi$指代一些描述我们的物理系统的东西,是个通称,而我们用拉格朗日量描述自由(即无相互作用)的场/粒子。
\item 我们只用三个Poincar\'e群双覆盖的基本表示,分别对应自旋$0, \tfrac{1}{2}$和$1$。没有基本粒子的自旋是$\tfrac{3}{2}$。
\end{itemize}

这是现代理论必须人为引入的假设。我们从实验上知道这些假设是正确的,但是目前为止没有更深刻的原理告诉我们为什么到$3$就停。

除此之外,还有两件事情没法从对称性中得到,但一个严谨的理论有时必须考虑它们:

\begin{itemize}
\item 我们只允许在拉格朗日量中引入尽可能低阶的非平庸微分算符$\partial_\mu$。一些理论使用一阶的微分算符$\partial_\mu$,而另一些理论中,Lorentz不变性禁止了一阶导数,从而二阶导数$\partial_\mu\partial^\mu$是最低阶的可能的非平庸项。除此之外我们就再也得不到一个合理的理论了。存在高阶导数项的理论没有下界,这导致能量可以是一个任意大小的负值;因此,这些理论中的态总可以变到能量更低的态,从而永远不会稳定。
\item 类似的,如果半整数自旋的粒子和整数自旋的粒子拥有完全一样的行为的话,宇宙中就不会有稳定的物质。因此,这两者必然有某些{\it 东西}不一样,而我们没得选,只有一种可能而且合理的选择\mpar{量子场论里一开始使用反对易子而不是对易子,从而防止理论没有下界。}是正确的。这引出了半整数自旋粒子的Fermi-Dirac统计的概念和整数自旋粒子的Bose-Einstein统计的概念。半整数自旋的粒子从而通常被称为Fermion\footnote{常译为费米子},它们中永远不存在两个粒子处在完全一样的态上。相反,这种情况对整数自旋的粒子 --- 通常被称为Boson\footnote{常译为玻色子} --- 是可能的。
\end{itemize}

最后呢,本书不能导出{\bf 引力}理论。大名鼎鼎的广义相对论就是优美而准确的描述引力的理论;然而这个理论与其他理论完全不一样,超出了本书的研究范围。而尝试将引力问题划入相同框架下的量子引力理论仍待完善:目前没有人能够成功得出它。但最后一章我们会对引力做一些评述。

\section[全书概览]{Book Overview 全书概览}\label{sec1.2}

\begin{center}
  \makebox[\textwidth][c]{\quad\quad\quad\quad
\small\xymatrix{
& \underset{\overset{|}{\text{不可约表示}}}{\text{Poincar\'e群的双覆盖}}\ar[d]  &\\
&\ar[dl]\ar[d]\ar[dr] &\\
(0,0): \text{自旋$0$表示}\ar[d]_{\text{作用}}^{\text{在}}&(\tfrac{1}{2},0)\oplus(0,\tfrac{1}{2}): \text{自旋$\tfrac{1}{2}$表示} \ar[d]_{\text{作用}}^{\text{在}}&(\tfrac{1}{2},\tfrac{1}{2}): \text{自旋$1$表示}\ar[d]_{\text{作用}}^{\text{在}}\\
\text{标量}\ar[d]_{\text{保证拉格朗日量}}^{\text{是(对称变换)不变的}}&\text{旋量} \ar[d]_{\text{保证拉格朗日量}}^{\text{是(对称变换)不变的}}&\text{矢量}\ar[d]_{\text{保证拉格朗日量}}^{\text{是(对称变换)不变的}}\\
\text{自由的自旋$0$体系的拉格朗日量}\ar[d]_{\text{欧拉-拉格朗日}}^{\text{方程}}&\text{自由的自旋$\tfrac{1}{2}$体系的拉格朗日量} \ar[d]_{\text{欧拉-拉格朗日}}^{\text{方程}}&\text{自由的自旋$1$体系的拉格朗日量}\ar[d]_{\text{欧拉-拉格朗日}}^{\text{方程}}\\
\text{Klein-Gordon方程}&\text{Dirac方程}&\text{Proka方程}
}}
\end{center}

这本书使用{\bf 自然单位制},也就是说Planck常数$h = 1$,光速$c=1$。这是基本理论的惯例,它免除了很多不必要的笔墨。而对于应用来说呢,这些常数需要被再一次的加上去从而回到标准的SI单位制。

{\bf 狭义相对论}的基本假设是我们的起始点;它们是:在所有的惯性参考系 --- 一些相互之间的速度保持恒定的参考系 --- 中,光的速度不变,为$c$;而且物理规律在在所有惯性系中相同。

满足这些对称性的所有的变换构成的集合叫做{\bf Poincar\'e群}。为了能够得到它们,我们接下来介绍一些数学知识从而使得我们能够利用好对称性。数学的这一分支叫群论。我们会得到Poincar\'e群\mpar{严格来说,我们会得到Poincar\'e群的双覆盖而不是Poincar\'e群本身。“双覆盖”这个词是由于一个群的双覆盖和这个群的关系是前者的两个元素与后者的一个元素相对应。这在后面的节\ref{sec3.3.1}中会讲到。}的不可约表示——你可以当成是组成其它所有表示的“地基”。这些表示就是后面用来描述粒子和场的不同自旋的表示。自旋一方面是对不同种类的粒子和场的标记,而另一方面也可以视为内禀角动量。

之后介绍{\bf 拉格朗日形式},它使得我们可以方便的在物理问题里面使用对称性。这里面的核心对象是{\bf 拉格朗日量},通过考虑不同物理系统的对称性来得到它\footnote{实际上,拉格朗日量是人为写出来的,而不是能够推导出来的。它是我们对某类体系『唯象』且经验的刻画 --- 译者注}。而在此之上我们就得到了{\bf 欧拉-拉格朗日方程},从而我们可以得到给定拉格朗日量的运动方程。利用Poincar\'e群的不可约表示,就可以导出带有不同自旋的场和粒子的基本运动方程。

这里面的中心思想是在Poincar\'e群中元素的的变换下,拉格朗日量必须是不变的。这使得在任意的参考系中得到的运动方程都是一样的,就像之前说的那样,“物理规律在在所有惯性系中相同”。

之后我们就会发现对于自旋$\tfrac{1}{2}$场的拉格朗日量的另一个对称性:在$\mathcal{U}(1)$变换下的对称性。类似的,对于自旋$1$的场的{\bf 内禀对称性}也可以找到。{\bf 局域}的$\mathcal{U}(1)$对称性可以使得我们得到自旋$\tfrac{1}{2}$场和自旋$1$场之间的{\bf 耦合项}。带这样的耦合项的拉格朗日量是{\bf 量子电动力学的拉格朗日量}的正确形式。类似的局域$\mathcal{SU}(2)$和$\mathcal{SU}(3)$变换会得到{\bf 弱和强相互作用的拉格朗日量}的正确形式。

作为补充,我们会讨论{\bf 自发对称破缺}和{\bf Higgs机制}。这些让我们能够去描述有质量的粒子\mpar{在对称性自发破缺前,拉格朗日量中的描述质量的项会因为破坏对称性而被禁止}。

之后我们导出了{\bf Noether定理},它给我们展示了对称性和守恒量之间深刻的联系。我们将物理量和相应对称性的生成元相等同。这就导出量子力学中最重要的方程\footnote{原书有typo}
\begin{align}\label{eq1.1}
[\hat{x}_i,\hat{p}_j]=i\delta_{ij}
\end{align}
和量子场论中
\begin{align}\label{eq1.2}
[\hat{\Phi}(x),\hat{\pi}(y)]=i\delta(x-y)
\end{align}

接下来通过对自旋$0$粒子的运动方程,即Klein-Gordon方程,取非相对论性\mpar{非相对论性指所有物体相比光速来说都运动的十分缓慢,从而狭义相对论的古怪的特性很小不会被观察到。}极限,得到了著名的{\bf Schrödinger方程}。这一点,同我们认识到的物理量与对称性生成元之间的联系,一起构成了{\bf 量子力学}的基石。

接下来我们从不同的运动方程\mpar{Klein-Gordon, Dirac, Proka和Maxwell方程}的解,和方程\eqref{eq1.2}出发来考察{\bf 自由场的量子场论}。这之后我们通过仔细审视包含不同自旋场之间的耦合的拉格朗日量来考虑相互作用。这使得我们可以讨论{\bf 散射过程的概率振幅}是如何得到的。

通过得到{\bf Ehrenfest定理},量子和{\bf 经典力学}的联系被我们展现出来。更进一步,我们够得了{\bf 经典电动力学}的基本方程,包括{\bf Maxwell方程组}和{\bf Lorentz力定律}。

最后简要的介绍现代{\bf 引力}理论 --- {\bf 广义相对论}的基本结构,并点出一些构建{\bf 引力的量子理论}的困难。

这本书的主要部分是关于处理对称性的数学工具,以及一种被称为{\bf 标准模型}的理论的导出。标准模型用量子场论来描述所有已知的基本粒子的行为。直到现在,所有标准模型的实验预言都是正确的。这里引入的其他任何理论都可以看作标准模型下的特殊情况,比如对于宏观物体:经典力学,或者对于低能的基本粒子:量子力学。对于那些从来没听说过现今已知的基本粒子和它们之间相互作用的读者,下一小节包含一个非常简短的概述。

\section[基本粒子和基本相互作用力]{Elementary Particles and Fundamental Forces 基本粒子和基本相互作用力}\label{sec1.3}

基本粒子分成两个主要的大类:{\bf bosons}和{\bf fermions}。{\bf Pauli不相容原理}说明不能有两个fermions处在完全相同的态上,而boson则可以有任意多个粒子处在同一个态上。这种自然界奇怪的事实导致了这些粒子截然不同的性质:

\begin{itemize}
\item Fermions构成物质\footnote{这里,『构成』实际上不是很合适的翻译。原文为`responsible for',直译的话应该叫Fermions对物质『负责』。不过大体上就是那么个意思吧。---译者注}
\item Bosons构成自然界的力
\end{itemize}

这说明,比如,原子是由fermions\mpar{原子有电子,质子和中子组成,它们都是fermions。然而注意质子和中子都不是最基本的粒子,它们由夸克组成,而夸克也是fermions。}组成的,但是电磁相互作用力是通过被称为光子的bosons传播的。这带来的一个最令人震惊的结果是竟然能形成稳定的物质。如果允许有无限多的fermions处在相同的状态,根本就不可能有稳定的物质存在,我们会在第\ref{chap6}中讨论这些。

目前我们知道四种基本作用力

\begin{itemize}
\item 电磁相互作用力,通过无质量的{\bf 光子}传播
\item 弱相互作用力,通过有质量的$\bf W^+, W^-$和$\bf Z$ - {\bf bosons}传播。
\item 强相互作用力通过无质量的{\bf 胶子}传播。
\item 引力,(或许)通过{\bf 引力子}传播。
\end{itemize}

这里面,一些相互作用对应的bosons是无质量的,而另一些不是,这揭露了一些自然界的深刻的事情。我们在做好足够的准备之后会去仔细理解这一点,现在只需要记住每一种相互作用力都与一个对称性相关。弱相互作用力传递的bosons是有质量的说明其对应的对称性被破坏了。这种{\bf 自发对称破缺}是{\bf 所有}基本粒子质量的起源。我们后面会看到这是通过与另一种基本boson,{\bf Higgs boson}的耦合来实现的。

携带有某种{\bf 荷}的基本粒子\mpar{在第\ref{chap7}章中我们会讲,所有的荷都有着一个相同的美丽起源}可以通过相应的某种力来相互作用。

\begin{itemize}
\item 对于电磁力来说,这种荷就是{\bf 电荷},因此只有带电粒子能够通过电磁力相互作用。
\item 对于弱相互作用力来说,这种荷称为\mpar{通常弱相互作用力的荷会伴随着“弱”这个前缀,所以也被称为弱同位旋;而且还存在一种其它的叫同位旋的概念适用于强相互作用里面的的复合物。不仅如此,这不是一个基本的荷,而本书中会省略“弱”这个前缀。}{\bf 同位旋}。所有已知的fermions都携带同位旋,因此它们之间通过弱作用力来相互作用。
\item 强作用力的荷叫{\bf 色},因为它与人对色彩的知觉有某种奇妙的共性。不要被这个名字搞糊涂,这个荷实际上与日常生活中的色彩没有任何关系。
\end{itemize}

基本的fermions被分成两个之类:{\bf 夸克},组成质子和中子的基本组成,与{\bf 轻子},如电子和中微子。其中的区别体现着夸克可以通过强作用力来相互作用,这就是说他们带色,而轻子则不是这样。夸克和轻子有三{\bf 代},每一代有两种粒子:

\begin{table}[h]
\centering
\begin{tabular}{c|c|c|c|c|c|c}
 \hline
 & 第一代 & 第二代 & 第三代 & 电荷 & 同位旋 & 色 \\
 \hline
 & 上(u)  & 粲(c) &  顶(t) & $+\tfrac{2}{3}e$  & $\tfrac{1}{2}$ & $\checkmark$\\
 夸克: & 下(d) & 奇异(s) &底(b) & $-\tfrac{1}{3}e$ & $-\tfrac{1}{2}$ & $\checkmark$\\
 \hline
 \ &\ &\ &\ &\ &\ &\ \\
 \hline
 & 电子中微子 & $\mu$子中微子 & $\tau$子中微子 & 0 & $+\tfrac{1}{2}$ & -\\
 轻子:&电子 & $\mu$子 & $\tau$子 & $-e$ & $-\tfrac{1}{2}$ & -\\
 \hline
\end{tabular}
\end{table}

一般的来说,不同的粒子可以通过{\bf 标签}来识别。除了荷和质量之外,它们还有很多其他的十分重要的标签,称为{\bf 自旋},你可以把它理解为内禀的角动量,我们会在第\ref{sec4.5.4}节中得到这一点。Bosons携带整数自旋,fermions携带半整数自旋。我们上面列出来的基本的fermions都只有$\tfrac{1}{2}$的自旋。基本的bosons有着$1$的自旋。除此之外,只有一种基本粒子有$0$自旋:Higgs boson。

每种粒子都有其反粒子,带着大小完全一样的相反\mpar{除了质量标签。这点目前正在接受实验的检验,比如瑞士日内瓦的CERN进行的ARGIS,ATRAP和ALPHA实验。}的标签。比如电子的反粒子就是正电子,但是一般的话我们不会认为的说一个新的名字,只是在原来的粒子前面加上前缀``反''。比如说,顶夸克的反粒子被称为反顶夸克。有些诸如光子\mpar{或许还有中微子,这一点目前正在接受各种关于搜索无中微子的双光子通道衰变实验研究。}一类的粒子自己就是自己的反粒子。

上面提到的这些想法之后都会更具体的解释。现在是时候去尝试推导理解这门能够正确描述粒子大观园中的不同特性是如何相互影响在一起的理论了。而第一步,也就是我们下一章的主题,是Einstein著名的狭义相对论。
