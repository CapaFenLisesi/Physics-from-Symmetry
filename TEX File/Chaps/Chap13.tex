%!TEX root = ..\main.tex
%!TEX encoding = UTF-8 Unicode

%——————————————————————————————————————————————————————————-
%	CHAPTER 13
%   Translator:SI
%——————————————————————————————————————————————————————————-
\chapterimage{f13.png} % Chapter heading image

\chapter{结束语}
\label{chap13}
在我看来,目前人类距离描述万物的万有理论还十分遥远。即使是本书主体部分所描述的优美理论也还存在着许多有待阐明的地方\footnote{比如本书多得要命的typo,赶快出修订版吧!——译者}。此外,如上一章所述,量子引力理论仍一筹莫展。

而且现在存在着实验证据(主要是宇宙学与天体粒子物理学的暗物质与暗能量疑难)表明现有理论仍有待改进。

我个人觉得现有疑难的背后仍蕴藏着大量内容,甚至或许是物理学的崭新框架。无论如何,未来的进展都非常有趣,衷心祝愿你会沿着这条路走向更远。