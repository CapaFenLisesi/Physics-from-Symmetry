%!TEX encoding = UTF-8 Unicode

%----------------------------------------------------------------------------------------
%	CHAPTER 4
%----------------------------------------------------------------------------------------

\chapterimage{chapter_head_1.pdf} % Chapter heading image

\chapter{The Framework (力学)框架/体系}
这一章的基本思路是,我们要在尽可能少的使用\emph{某些东西}的前提下,得到正确的关于自然的方程。\emph{某些东西}是什么?有一件事是确定的:它不应该在 Lorentz 变换下改变,否则我们会在不同的参考系下得到不同的自然规律。在数学意义上,它意味着我们寻找的这个东西是个标量,依照洛伦兹群的 \( (0,0) \) 表示作变换。再考虑到自然总依简单而行,我们已经足够导出关于自然的方程了。
%flag The equations of nature
%( 0,0 ) representation  \( (0,0) \) 表示
%the system in question
从这个想法出发,我们将会引入{\bf 拉格朗日形式(Lagrangian formalism)}。通过极小化理论的中心对象,我们可以得到用以描述问题中的物理系统的运动方程。极小化过程的结果被称作 {\bf Euler-Lagrange 方程}。

通过拉氏形式,我们可以得到物理中最重要的定理:Noether 定理。这个定理揭示了对称性和守恒量
\footnote{守恒量指的是不随时间变化的物理量。例如一个给定体系的能量或动量。数学上意味着\(\partial_t Q = 0 \rightarrow Q = \text{常数}\)}。
之间的深刻联系。我们将在下一章中利用它来理解,理论是如何来描述实验测量量的。

\section{Lagrangian Formalism 拉氏形式}

拉氏形式是在基础物理中被广泛运用的一个强有力的框架
\footnote{物理中当然有其他框架,例如以{\bf 哈密顿量(Hamiltonian)}为中心对象的{\bf 哈密顿形式(Hamiltonian formalism)}。哈密顿量的问题在于它不是洛伦兹不变的,因为它所代表的能量,仅仅是{\bf 四动量(covariant energy-impulse vector)}的一个分量}
。由于理论的基本对象--{\bf 拉格朗日量(Lagrangian)}是一个标量
\footnote{标量指依照洛伦兹群的 \( (0,0) \) 表示作变换的对象。这意味着它不在洛伦兹变换下改变}
,它相对简单。如果你希望从对称性的观点考虑问题,这种形式将会是非常有用的。如果我们要求拉氏量的积分,{\bf 作用量(action)},在某些对称变换下不变,我们即要求体系的动力学遵从该对称性。

\subsection{Fermat 原理}
\begin{quote}
Whenever any action occurs in nature, the quantity of action employed by this change is the least possible.\\
- Pierre de Maupertius \footnote{Recherche des loix du mouvement (1746)}
\end{quote}

拉氏形式的思想源于 Fermat 原理:光在两空间点间传播总依耗时最短的路径\(q(t)\)而行。数学上来讲,如果我们定义给定路径\(q(t)\)的作用量为
\begin{equation*}
S_{\text{light}}[{\mathbf q}(t)] = \int ~{\mathrm d}t
\end{equation*}
而我们的任务便是找到一条特定的路径\(q(t)\)使作用量取极小值
\footnote{此处的作用量仅仅是沿给定路径对时间的积分,但一般而言作用量会更加复杂,我们待会儿就能见到}
为了得到一个给定{\bf 函数}的极小值
\footnote{一般而言,我们希望找到{\bf 最值(extremums)},即最小值和最大值。}
,我们可以求得其导函数并令其为零;而为了找到{\bf 泛函}\(S[q(t)]\)——函数\(q(t)\)的函数\(S\)——的极小值,就得要一个新的数学工具:{\bf 变分法}。

\subsection{变分法:基本思想}

在思考如何发展一套能够找到泛函极值的新理论之前,我们需要倒回去想想什么给出一个数学上的极小点。变分法给出的答案是,极小点由极小点邻域的性质决定。例如,让我们尝试寻找一个寻常函数\(f(x)=3x^2+x\)的极小点\(x_{\min}\)。我们从一个特定点\(x=a\)出发,仔细考察其邻域。数学上它意味着\(a + \epsilon\),其中\(\epsilon\)代表无穷小量(可正可负)。我们将\(a\)的变分代入函数\(f(x)\):
\[
f(a+\epsilon) = 3(a+\epsilon)^2+(a+\epsilon) = 3(a^2+2a\epsilon+\epsilon^2)+a+\epsilon \text{。}
\]
如果\(a\)是极小点,\(\epsilon\)的一阶变分必需为零,否则我们可以取\(\epsilon\)为负\(\epsilon < 0\),这样\(f(a+\epsilon)\)就会比\(f(a)\)更小\footnote{\textcolor{red}{译注:此处讨论有误}}。因此,我们将线性依赖于\(\epsilon\)的项取出并令其为零。
\[
3 \cdot 2a\epsilon + \epsilon \overset{\text{!}}{=} 0 \leftarrow 6a+1\overset{\text{!}}{=}0\text{。}
\]
由此我们找到极小点
\[
x_{\min} = a = -\frac{1}{6} \text{,}
\]
