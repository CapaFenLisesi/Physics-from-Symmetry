%!TEX root = ..\main.tex
%!TEX encoding = UTF-8 Unicode

%——————————————————————————————————————————————————————————-
%   CHAPTER 11
%   Translator:SI
%   Proofread : lh1962
%——————————————————————————————————————————————————————————-
\chapterimage{chapter_head_1.pdf} % Chapter heading image


\chapter[电动力学]{Electrodynamics \quad 电动力学}
\label{chap11}
第\ref{chap7}章已经导出了经典电动力学最重要的方程 --- 非齐次Maxwell方程(组)\eqref{equ7.22}式:
\begin{equation}
\label{equ11.1}
    \partial_\sigma (\partial^\sigma A^\rho - \partial^\rho A^\sigma ) = J_\rho
\end{equation}
利用电磁张量\mpar{电磁张量\\ $F^{\sigma \rho} \equiv (\partial^\sigma A^\rho - \partial^\rho A^\sigma)$.}可将上式写为更紧凑的形式:
\begin{equation}
\label{equ11.2}
    \partial_\sigma F^{\sigma \rho} = J^\rho
\end{equation}
\ref{sec7.1.6}节导出了$J_\rho$为Noether流,即$\partial_\rho J^\rho = 0$. 这一守恒流即为宏观理论中的四维电流。电磁张量$F^{\sigma \rho}$是反称张量($F^{\sigma \rho} = -F^{\rho \sigma}$),这可以从定义式$F^{\sigma \rho} \equiv \partial^\sigma A^\rho - \partial^\rho A^\sigma$直接看出,因此$F^{\sigma \rho}$只有$6$个独立分量,其中三个是
\begin{equation}
\label{equ11.3}
    F^{i0} = \partial^{i} A^0 - \partial^0 A^i
\end{equation}
另外三个是
\begin{equation}
\label{equ11.4}
    F^{ij} = \partial^i A^j - \partial^j A^i = (\delta^i_\ell \delta^j_m - \delta^i_m \delta^j_\ell) \partial^\ell A^m \underbrace{=}_{\mathclap{ \delta^i_\ell \delta^j_m - \delta^i_m \delta^j_\ell = \epsilon^{ijk} \epsilon^{k\ell m}, \text{证明作为习题} }} \epsilon^{ijk} \epsilon^{k\ell m} \partial_\ell A^m
\end{equation}
这些独立分量通常记为:
\begin{align}
\label{equ11.5}
    \partial^i A^0 - \partial^0 A^i &\equiv E^i \\
\label{equ11.6}
    \epsilon^{ijk} \partial^j A^k  &\equiv -B^i
\end{align}
因此
\begin{align}
\label{equ11.7}
    F^{i0} &= E^i \\
\label{equ11.8}
    F^{ij} &= \epsilon^{ijk} \epsilon^{k\ell m} \partial^\ell A^m = -\epsilon^{ijk} B^k
\end{align}
将非齐次Maxwell方程组\mpar{\eqref{equ11.2}式称为方程组是因为它代表了$\rho = 0, 1, 2, 3$一共四个方程。}重新写为:
\begin{equation}
\label{equ11.9}
    \partial_\sigma F^{\rho \sigma} = \partial_0 F^{\rho 0} - \partial_k F^{\rho k} = J^\rho
\end{equation}
对于三个空间分量($\rho \to i$)有\mpar{$\epsilon^{jk\ell} \partial_k B^\ell = \big( \nabla \times \vec{B} \big)^j$,后者就是向量分析里面的旋度运算(nabla算子叉乘向量)。}:
\begin{align}
    &\partial_0 F^{i0} - \partial_k F^{ik} = \partial_0 E^i + \epsilon^{ik\ell} \partial_k B^\ell = J_i \notag \\
\label{equ11.10}
    &\to \partial_t \vec{E} + \nabla \times \vec{B} = \vec{J}
\end{align}
对于时间分量($\rho = 0$分量):
\begin{align}
   & \partial_0 \underbrace{F^{00}}_{\mathclap{ =0, \text{见} F^{\rho \sigma} \text{定义式} }} - \partial_k F^{0k} &\overbrace{=}^{F^{\mu \nu} = -F^{\nu \mu}} \partial_k F^{k0} \underbrace{=}_{\eqref{equ11.7} \text{式}} \partial_k E^k = J^0 \notag \\
\label{equ11.11}
   & \to \nabla \cdot \vec{E} &= J_0
\end{align}
这是非齐次 Maxwell 方程组在工程等领域中的常用形式。

\section[齐次 Maxwell 方程组]{The Homogeneous Maxwell Equations \quad 齐次Maxwell方程组}
\label{sec11.1}
定义{\bf 对偶(dual)电磁张量}$\tilde{F}^{\mu \nu}$为四维Levi-Civita符号%
\mpar{四维Levi-Civita符号$\epsilon^{\mu \nu \rho \sigma}$的定义见附录\ref{appendix.B.5.5}。}%
与电磁张量的缩并:
\begin{equation}
\label{equ11.12}
    \tilde{F}^{\mu \nu} = \epsilon^{\mu \nu \rho \sigma} F^{\rho \sigma}
\end{equation}
从$F^{\mu \nu}$的定义可以导出,$F^{\mu \nu}$与任意全反称张量(比如$\epsilon^{\mu \nu \rho \sigma}$)缩并后的导数为零,即:
\begin{equation}
\label{equ11.13}
    \partial_\mu \tilde{F}^{\mu \nu} = \partial_\mu \epsilon^{\mu \nu \rho \sigma} (\partial_\sigma A_\rho - \partial_\rho A_\sigma) = 0
\end{equation}
这一结论实际是如下命题的特殊情形:两个对称指标与两个反对称指标缩并\footnote{例如,张量$A^{\alpha \beta \mu \nu} = A^{\alpha \beta \nu \mu}$, $\mu, \nu$是一对对称指标;张量$B_{\rho \sigma \eta \delta} = -B_{\sigma \rho \eta \delta}$, $\rho, \sigma$是一对反称指标。一对对称指标与反称指标的缩并结果必为零(即所得张量分量均为$0$):$C^{\alpha \beta}_{\eta \delta} \equiv A^{\alpha \beta \gamma \varphi} B_{\gamma \varphi \eta \delta} = 0.$}所得结果为零%
\mpar{详见附录\ref{appendix.B.5.4}。}。%
以\eqref{equ11.13}式括号内的第一项为例:
\begin{align}
    \epsilon^{\mu \nu \rho \sigma} \partial_\mu \partial_\sigma A_\rho &= \frac{1}{2} ( \epsilon^{\mu \nu \rho \sigma} \partial_\mu \partial_\sigma A_\rho + \epsilon^{\mu \nu \rho \sigma} \partial_\mu \partial_\sigma A_\rho) \notag \\
    & \underbrace{=}_{\mathclap{ \text{重命名哑指标} }} \frac{1}{2} ( \epsilon^{\mu \nu \rho \sigma} \partial_\mu \partial_\sigma A_\rho + \epsilon^{\sigma \nu \rho \mu} \partial_\sigma \partial_\mu A_\rho) \notag \\
\label{equ11.14}
    & \underbrace{=}_{\mathclap{ \epsilon^{\mu \nu \rho \sigma} = -\epsilon^{\sigma \nu \rho \mu}, \partial_\mu \partial_\sigma = \partial_\sigma \partial_\mu } } \frac{1}{2} (\epsilon^{\mu \nu \rho \sigma} \partial_\mu \partial_\sigma A_\rho - \epsilon^{\mu \nu \rho \sigma} \partial_\mu \partial_\sigma A_\rho) = 0 \quad \surd
\end{align}
同理可导出括号中的第二项为零。方程组%
\mpar{下式代表$\nu = 0, 1, 2, 3$四个方程组成的方程组。}%
\begin{equation}
\label{equ11.15}
    \partial_\mu \tilde{F}^{\mu \nu} = 0
\end{equation}
称为{\bf 齐次Maxwell方程组(homogeneous Maxwell equations)}。从推导过程可见它源自电磁张量$F^{\mu \nu}$的定义。下面将它写成用电场$\vec{E}$与磁场$\vec{B}$表示的形式,考虑$\nu = 0$对应的方程:
\begin{align}
    0 &= \partial_\mu \tilde{F}^{\mu 0} \notag \\
    &= \partial_\mu \epsilon^{\mu 0 \rho \sigma} F^{\rho \sigma} \notag \\
    &= \partial_0 \underbrace{\epsilon^{00 \rho \sigma}}_{= 0} F^{\rho \sigma} - \partial_i \epsilon^{i0\rho \sigma} F^{\rho \sigma} \notag \\
    & \underbrace{=}_{\mathclap{ \epsilon^{i 0 \rho \sigma} = 0, \text{对于} \rho = 0 \text{或} \sigma = 0 }} -\partial_i \epsilon^{i0jk} F^{jk} \notag \\
    &= \partial_i \epsilon^{0ijk} F^{jk} \notag \\
    &\underbrace{=}_{ \eqref{equ11.8} \text{式} } -\partial_i \underbrace{\epsilon^{0ijk} \epsilon^{\ell jk} }_{= 2\delta^{i \ell}} B^\ell \notag \\
    &= -2 \partial_i \delta_{i\ell} B^\ell = -2 \partial_i B^i \notag \\
\label{equ11.16}
    \Rightarrow & \partial_i B^i = 0, \text{用向量记号表示:} \nabla \cdot \vec{B} = 0
\end{align}
考虑空间分量:$\nu = i$对应的方程可以导出:
\begin{equation}
\label{equ11.17}
    \nabla \times \vec{E} + \partial_t \vec{B} = 0
\end{equation}
这是齐次 Maxwell 方程组在应用领域中的常见形式。

\section[Lorentz力]{The Lorentz Force \quad Lorentz力}
\label{sec11.2}
利用上一章导出的量子力学与经典力学的联系(Ehrenfest定理)可以推导出著名的Lorentz力公式。考虑在电磁场中运动的非相对论性且无自旋的粒子,其Schr\"{o}dinger方程包含电磁场的耦合项(\eqref{equ8.23}式)%
\mpar{该式是从与外电磁场耦合的Klein-Gordon方程导出,亦即,从\ 与自旋$1$无质量场(电磁场)耦合的自旋$0$的粒子的拉格朗日量导出的。因此推导的出发点还是用来导出拉格朗日量的Lorentz对称性与规范对称性。}%:
\begin{equation}
\label{equ11.18}
    i \partial_t \Psi = \underbrace{ \left( \frac{1}{2m} (\vec{p} - q\vec{A})^2 + q \Phi \right)}_{\equiv H} \Psi
\end{equation}
定义该系统的动量%
\mpar{这个定义来自Noether定理:系统的平移不变性对应守恒量-动量,$\frac{\partial L}{\partial \dot{x}} = \Pi$.}%
为$\vec{\Pi} = \vec{p} - q \vec{A}$, 则系统的哈密顿量为:
\begin{equation*}
    H = \frac{1}{2m} |\vec{\Pi}|^2 + q \Phi
\end{equation*}
有了哈密顿量,再重复与上一章相同的过程,就能得到类似\eqref{equ10.1}式的结果。因为这里涉及到依赖于时间的算符,因而推导中包含对时间的偏导数项:
\begin{align*}
    \frac{d}{dt} \langle \hat{O} \rangle &= \frac{1}{i} \langle [O, H] \rangle + \langle \frac{\partial \hat{O}}{\partial t} \rangle \\
    \to \frac{d}{dt} \langle \vec{\Pi} \rangle &= \frac{1}{i} \langle [\vec{\Pi}, H] \rangle + \langle \frac{\partial \vec{\Pi}}{\partial t} \rangle,
\end{align*}
$\partial \vec{\Pi} / \partial t \neq 0$, 因为$\vec{A}$显含时间。将$H$的具体形式带入得:
\begin{align*}
    \to \frac{d}{dt} \langle \vec{\Pi} \rangle &= \frac{1}{i} \langle [\vec{\Pi}, \frac{1}{2m} |\vec{\Pi}|^2 + q \Phi] \rangle + \langle \frac{\partial \vec{\Pi}}{\partial t} \rangle \\
    \to \frac{d}{dt} \langle \vec{\Pi} \rangle &= \frac{1}{i} \langle [\vec{\Pi}, \frac{1}{2m} |\vec{\Pi}|^2 \rangle + \underbrace{ \frac{1}{i} \langle [\vec{\Pi}, q\Phi] \rangle }_{ = \langle q \nabla \Phi \rangle } + \langle \frac{\partial \vec{\Pi}}{\partial t} \rangle .
\end{align*}
最后一步中$[\vec{\Pi}, q\Phi]$的推导可类比于上一章中$[\hat{p}, V]$的推导(因为$[\vec{A}, \Phi] = 0$)。接下来计算$[|\vec{\Pi}|^2, \vec{\Pi}]$,它不等于零,因为各分量$\Pi_i$之间不对易:
\begin{equation}
\label{equ11.19}
    [\Pi_i, \Pi_j] = -\frac{q}{i} \left( \frac{\partial A_j}{\partial x_i} - \frac{\partial A_i}{\partial x_j} \right) = -\frac{q}{i} \epsilon_{ijk} \underbrace{B_k}_{\mathclap{= \epsilon_{k\ell m}\frac{\partial}{\partial x_\ell} A_m }}
\end{equation}
上面利用了磁场定义$\vec{B} = \nabla \times \vec{A}$的指标形式。粒子的速度%
\mpar{(Newton极限下)动量除以质量即为速度:$p = mv$。}%
定义为$\vec{v} \equiv \vec{\Pi} / m$,由此可得:
\begin{align*}
    \frac{d}{dt} \langle \vec{\Pi} \rangle = \langle q \nabla \Phi \rangle - \langle q (\vec{v} \times \vec{B}) \rangle +  \underbrace{\left\langle \frac{\partial \vec{\Pi}}{\partial t} \right\rangle}_{= -q \frac{\partial \vec{A}}{\partial t}} \\
    \frac{d}{dt} = -q \langle (\vec{v} \times \vec{B}) \rangle + q \underbrace{\left\langle \nabla \Phi - \frac{\partial \vec{A}}{\partial t} \right\rangle}_{ = \langle E \rangle, \text{见}\eqref{equ11.5}\text{式} }
\end{align*}
最终得到
\begin{equation}
\label{equ11.20}
    \frac{d}{dt} \langle \vec{\Pi} \rangle \equiv F_{\text{Lorentz}} = -q \big( \langle \vec{v} \times \vec{B} \rangle + \langle E \rangle \big).
\end{equation}
上式即为带电粒子在外电磁场中的运动方程,其解即为粒子轨迹。

\section[Coulomb 势]{Coulomb Potential \quad Coulomb 势}
\label{sec11.3}
第\ref{chap7}章讲过拉格朗日量具有内禀对称性,可以利用这一自由度简化计算,也就是说将场进行内禀变换以使计算大大简化。只要进行规范变换(使拉格朗日量不变),则变换前后的场蕴含的物理规律是相同的。通常选择的规范是关于光子场$A^\mu$的,规定%
\mpar{可以证明这一规范条件的存在性(略),相关细节可参见电动力学教材。}%
$\partial_\mu A^\mu = 0$,这称为 Lorentz 规范。利用这一规范条件将非齐次 Maxwell 方程组简化为:
\begin{equation}
\label{equ11.21}
    \partial_\sigma (\partial^\sigma A^\rho - \partial^\rho A^\sigma) = \partial_\sigma \partial^\sigma A^\rho = J^\rho
\end{equation}
考虑静电情况,静止的电荷关于坐标系原点对称分布。求解外部区域(无源,即$J\rho = 0$)的电磁场,外部区域的Maxwell方程组为\footnote{原文下式等号右侧上下标有误,译文已更正。}:
\begin{equation}
\label{equ11.22}
    \partial_\sigma \partial^\sigma A^\rho = \partial_0 \partial^0 A^\rho - \partial_i \partial^i A^\rho = 0.
\end{equation}
考虑到系统是静态($\partial_0 A^\rho = 0$)、对称的,因此将方程用球坐标表示%
\mpar{除了用直角坐标$x, y, z$表示物体的位置外,还可用球坐标$r, \theta, \phi$表示。 $\partial_i \partial^i A^\rho = \frac{\partial^2}{\partial r^2} (r A^\rho) + \frac{1}{r^2 \sin \theta} \frac{\partial}{\partial \theta} \left( \sin \theta \frac{\partial A^\rho}{\partial \theta} \right) + \frac{1}{r^2 \sin^2 \theta} \frac{\partial^2 A^\rho}{\partial \phi^2}$. 当系统具有球对称性时选择球坐标十分方便,因为物理量只与$r$(到原点的距离)有关。}%
。这样可以忽略除了包含$\partial_r$之外的所有项%
\mpar{对于球对称性的场,$\partial_\theta A = \partial_\phi A = 0$.}%
,由此可得:
\begin{equation}
\label{equ11.23}
    \to \frac{\partial^2}{\partial r^2} (r A^\mu) = 0.
\end{equation}
上式具有通解:
\begin{equation}
\label{equ11.24}
    A^\mu = \epsilon^\mu \frac{C}{r} + \epsilon^\mu D
\end{equation}
其中$\epsilon^\mu$表示某个常数4-向量,$C, D$表示常数。场$A^\mu$在无穷远处为零,故$D = 0$,$A^\mu$的$0$分量即为著名的Coulomb势:
\begin{equation}
\label{equ11.25}
    A^0 = \Phi = \frac{C}{r} = \frac{Ze}{r},
\end{equation}
其中$Z$为整数,$e$为元电荷(电子电荷)。由于电子电荷是量子化的,故常数$C$写为$Ze$的形式。标准模型不能对电荷量子化做出满意的解释。

\section*{Further Reading Tips \quad 阅读建议}
\begin{itemize}
    \item {\bf Richard P. Feynman - The Feynman Lectures on Physics Volume 2}%
    \mpar{Richard P. Feynman, Robert B. Leighton, and Matthew Sands. The Feynman Lectures on Physics: Volume 2. Addison-Wesley, 1st edition, 2 1977. ISBN 9780201021172}\footnote{译注:此书新版可在此免费在线查看: \url{http://www.feynmanlectures.caltech.edu/II_toc.html}。\\另有新版中译本:费恩曼,莱顿,桑兹著;李洪芳,王子辅,钟万蘅译.费恩曼物理学讲义,新千年版,第三卷,\\上海:上海科学技术出版社,4 2013,ISBN 9787547816370。} %
    是不错的电动力学入门教材。
    \item {\bf David J. Griffiths - Introduction to Electrodynamics}%
    \mpar{David J. Griffiths. Introduction to Electrodynamics. Addison-Wesley, 4th edition, 10 2012. ISBN 9780321856562}\footnote{译注:国内有由机械工业出版社出版的影印本和中译本,\\影印本ISBN为9787506272896,贾瑜等的中译本ISBN为9787111444046。}%
    是更详细的电动力学优秀入门教材。
\end{itemize}
