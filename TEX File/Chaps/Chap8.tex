%!TEX encoding = UTF-8 Unicode

%----------------------------------------------------------------------------------------
%	CHAPTER 8
%   translator: laserdog
%----------------------------------------------------------------------------------------

\chapterimage{chapter_head_1.pdf} % Chapter heading image

\chapter[量子力学]{Quantum Mechanics 量子力学}\label{chap8}

{\Huge\bf 总结\\ \ \\}
在这章中,我们将讲量子力学。这里的每一件事情的基础是我们第\ref{chap5}章做的那种对应。其结果就是我们可以得到{\bf 先对论性能量-动量关系}。

%而我们之前已经讨论过了量子的框架是如何运作的;这实际上只是对最简单的标量型粒子做运动方程,之后我们就对Klein-Gordon方程做了非相对论性极限,得到了著名的{\bf Schrödinger方程}。
在讨论过了量子的框架是如何运作的之后,我们就对Klein-Gordon方程做了非相对论性极限,得到了著名的{\bf Schrödinger方程}。这样做是因为这个方程实际上只是对最简单的标量型粒子做的运动方程。
方程的解被解释为概率振幅,并且我们随后用{\bf 波动力学}的方法分析了两个简单的例子。

之后呢,我们引入了{Dirac符号},这对于我们理解量子力学的结构十分有帮助。系统的初始态被一个抽象的态矢量$|i\rangle$标记,我们乘它为{\bf 右矢}。测量这个初始态变成某个特定的末态的概率振幅可以通过形式上作用一个我们称之为{\bf 左矢},记为$\langle f|$的东西上去。左矢和右矢在一起会形成一个复数,我们解释为过程$i\to f$的概率振幅$A$。发生这个过程的概率就是$|A|^2$。之后我们会讲{\bf 投影算符}。我们会看它是如何与{\bf 完备性关系}一起用来把任意一个态用任意一个算符的本征态来进行展开的。我们之前使用的波动力学的角度可以看成是一个特例,我们用坐标来进行的展开。在Dirac符号下,Schrödinger方程是被用来计算态的时间演化的。为了明确的建立这种联系,我们用Dirac符号来会看一个我们已经用波动力学解过的例子。

\section[]{Particle Theory Identifications }

