%!TEX encoding = UTF-8 Unicode

%----------------------------------------------------------------------------------------
%   CHAPTER 8
%   translator: laserdog, 日始之音
%   proofreader: SI
%----------------------------------------------------------------------------------------
\def\pmm{\begin{pmatrix}}
\def\pmme{\end{pmatrix}}

\chapterimage{f08.png} % Chapter heading image

\chapter[量子力学]{Quantum Mechanics \quad 量子力学}\label{chap8}

{\Huge\bf 总结\\ \ \\}
本章讲述量子力学,基于第\ref{chap5}章讲过的对应关系\footnote{物理量$\to$ 对称性生成元,见\ref{equ5.1}式。},由此首先可导出{\bf 相对论性能量-动量关系}。

在建立量子力学的基本框架后,我们就对Klein-Gordon方程(最简单的标量型粒子的运动方程)取非相对论极限,得到著名的{\bf Schrödinger方程},方程解被解释为概率幅,随后我们用{\bf 波动力学}的方法分析两个简单例子。

之后呢,我们引入了{\bf Dirac符号},这对于理解量子力学的结构十分有帮助。系统的初始态被一个抽象的态矢量$|i\rangle$标记,称之为{\bf 右矢}。测量该初始态得到某一特定末状态的概率幅可以用左矢(记作$\langle f|$,表示末态)乘以右矢$|i \rangle$计算。左矢与右矢的乘积$\langle f | i \rangle$是一个复数,即为状态$i \to f$的概率幅,发生这个过程的概率就是$|A|^2$。之后讨论{\bf 投影算符}。我们会看到它是如何与{\bf 完备性关系}一起用来把任意状态用任意一个算符的本征态来进行展开的。之前常用的波动力学方法可以视为将态矢在坐标基底展开的特例。在Dirac符号下,Schrödinger方程用来计算态的时间演化。为了阐明这一联系,我们会用Dirac符号来重新讨论一个前面用波动力学求解过的例子。

\section[对应到粒子理论]{Particle Theory Identifications \quad 对应到粒子理论}\label{sec8.1}
前面导出的方程\mpar{Klein-Gordon, Dirac, Proka, Maxwell方程}可以用在粒子理论和场理论中。本章考虑它们应用于粒子理论的情形。因此动力学变量就是粒子(们)的坐标、能量和动量。第\ref{chap5}章讲过,这些物理量可视为相应对称性的生成元:\mpar{见\eqref{equ3.240},\eqref{equ3.244}和第\ref{chap5}章。}
\begin{itemize}
\item 动量$\hat{p}_i=-i\partial_i$\\
\item 坐标$\hat{x}_i=x_i$\\
\item 能量$\hat{E}=i\partial_0$\\
\item 角动量$\hat{L}_i=i\frac{1}{2}\epsilon_{ijk}(x^j\partial^k-x^k\partial^j)$
\end{itemize}

在讨论这些算符是如何应用于量子力学之前,我们先用它们导出现代物理最重要的方程之一。
\section[相对论性能量-动量关系]{Relativistic Energy-Momentum Relation \quad 相对论性能量-动量关系}\label{sec8.2}
第\ref{sec6.2}节推导了自旋$0$自由场的运动方程——Klein-Gordon方程:
\[(\partial_\mu\partial^\mu+m^2)\Phi=0, \]

利用上一节的那些对应关系\mpar{$p_\mu=\left(\begin{matrix}p_0\\p_1\\p_2\\p_3\end{matrix}\right)=\left(\begin{matrix}p_0\\\vec{p}\end{matrix}\right)=\left(\begin{matrix}E\\\vec{p}\end{matrix}\right)$},
\begin{align}
  (\partial_\mu\partial^\mu+m^2)\Phi&=(\partial_0\partial_0-\partial_i\partial_i+m^2)\Phi \notag \\
  &=\left(\left(\frac{1}{i}E\right)\left(\frac{1}{i}E\right)-\left(-\frac{1}{i}p_i\right)\left(-\frac{1}{i}p_i\right)+m^2\right)\Phi \notag \\
\label{equ8.1}
  &=(-E^2+\vec{p}^{\,2}+m^2)\Phi=0 \\
\label{equ8.2}
  \to\ E^2=\vec{p}^{\,2}+m^2 & \quad\text{ 或者用4-矢量表示:}p_\mu p^\mu=m^2
\end{align}
这就是著名的狭义相对论{\bf 能量-动量关系}。上式应用于静止粒子($\vec{p} = 0$)即得到Einstein著名的质能方程:
\[E^2=m^2\to E=mc^2 \]
其中为了明确起见我们把$c^2$写回来。现在就理解了Poincare群的第一个Casimir算符$p_\mu p^\mu$的标量值(见\ref{equ3.258}式)为何选为$m^2$了。$p_\mu p^\mu$就代表着粒子质量的平方,而通过实验,比如说测量粒子的能量和动量,就可测量质量$m=\sqrt{E^2-\vec{p}^{\,2}}$。同理,我们也可以理解为什么\ref{sec6.2}节要把拉格朗日量里面的系数常数记作$m^2$。

\section[量子力学的数学形式]{The Quantum Formalism\quad 量子力学的数学形式}
\label{sec8.3}

前面说过,物理量用算符表示,这些算符作用于什么对象?首先需要注意每个算符都有一组本征函数,就像矩阵的本征向量一样。矩阵是有限维的,因此矩阵有有限多个本征向量;而算符通常作用在无限维的矢量空间中,而得到无穷多个本征函数。比如,动量算符的本征函数满足的本征方程为
\begin{align}
\label{equ8.3}
	\underbrace{-i\partial_i\Psi}_{\mathclap{\text{算符}}}=\underbrace{p_i}_{\mathclap{\text{本征值}}}\overbrace{\Psi}^{\mathclap{\text{本征函数}}}
\end{align}
其中,$p_i$是一个数。一个显然的解为
\begin{align}
	&\underbrace{C}_{\mathclap{=\text{常数}}}e^{ip_ix_i} \notag \\
\label{equ8.4}
	&\to-i\partial_iCe^{ip_ix_i}=p_iCe^{ip_ix_i}\checkmark
\end{align}
但要注意,对于任意的$p_i$,这都是一个解。因此我们发现有无穷多个动量算符$\hat{p}_i=-i\partial_i$的本征函数。能量本征函数同理
\begin{align}
i\partial_0\Phi=E\Phi
\end{align}
或者角动量的本征函数\mpar{角动量本征函数的求解更加复杂,因为角动量算符比其它算符本身就更复杂。我们找不到一组三个角动量分量共同的本征态,因为$[\hat{L}_i,\hat{L}_j]\neq0$。这在后面就会详细讨论,最终选取的本征函数是角动量第三个方向$\hat{L}_3$(和角动量算符的平方$\hat{L}^2$,它与其它的角动量分量都对易$[\hat{L}^2,\hat{L}_j]=0$)的完备归一基,即著名的{\bf 球谐函数基}。}。类比于矩阵的本征矢量,这些本征函数可以看做基\mpar{在矩阵中,本征矢量是矩阵对应的矢量空间的基},而这就意味着我们可以把任意的函数$\Psi$按照这些基进行展开。例如,在动量本征函数\mpar{注意,这其实就是我们在附录\ref{appendixD.1}中介绍的Fourier变换,因子$\frac{1}{\sqrt{2\pi}}$是出于惯例的考虑。}进行展开(简明起见,考虑一维情形)
\begin{align}
\label{equ8.6}
	\Psi=\frac{1}{\sqrt{2\pi}}\int_{-\infty}^\infty dp\Psi_pe^{-ipx}
\end{align}
其中,$\Psi_p$是展开系数,类似于矢量$\vec{v}=v_1\vec{e}_1+v_2\vec{e}_2+v_3\vec{e}_3$中的$v_1,v_2,v_3$。对于一些系统来说,我们的边界条件使得最后得到的基是离散而不连续的。我们可以,比如说,用能量本征态$\Phi_{E_c}$进行展开:
\begin{align}
\label{equ8.7}
	\Psi=\sum_n c_n\Phi_{E_n}
\end{align}
注意,一般来说,一个算符的本征函数并不是另一个算符的本征函数。只有当两个算符对易,即$[A,B]=AB-BA=0$时,才可以找到一组同时是两个算符本征函数的基。下面证明这个命题。设$[C,D]\neq0\to CD\neq DC$,而对于$C$的某个本征函数$\Psi$,有$C\Psi=c\Psi$,其中$c$是它的本征值。如果$\Psi$同时也是$D$的本征函数$D\Psi=d\Psi$,则有
\[CD\Psi=Cd\Psi\underbrace{=}_{\mathclap{\text{因为$d$只是一个数}}}dC\Psi=dc\Psi \]
\[DC\Psi=Dc\Psi=cD\Psi=cd\Psi\underbrace{=}_{\mathclap{\text{因为数之间互相对易}}}dc\Psi \]
\begin{align}
\label{equ8.8}
	\to DC=CD\quad \text{这与$[C,D]\neq0$矛盾,因而不存在共同的本征函数。}
\end{align}

一般来说,物理量算符作用在描述物理系统的{\bf 态}\mpar{这里使用$\Psi$是出于量子力学的惯例,尽管前面只用它表示描述自旋$0$粒子的旋量。}$\Psi$上,通过求解运动方程得到$\Psi$。

一般的解按照某组基展开都会超过一项。举个例子,考虑两个能量本征态\footnote{即能量算符(蛤密顿算符)$\hat{H}$的本征态,能量本征值为$E$的本征态记作$\Phi_E$.}$\Psi = c_1 \Phi_{E_1} + c_2 \Phi_{E_2}$\mpar{这意味着所有的展开系数$\Psi=\sum_nc_n\Phi_{E_n}$是零。},能量算符作用于它可得
\begin{align}
\label{equ8.9}
	\hat{E}\Psi = \hat{E}(c_1\Phi_{E_1} + c_2\Phi_{E_2})=c_1E_1\Phi_{E_1}+c_2E_2\Phi_{E_2}\neq E(c_1\Phi_{E_1}+c_2\Phi_{E_2})
\end{align}
由此可见不同能量(的能量本征态)形成的{\bf 叠加态}一般不是能量算符的本征态,因为本征态需要满足定义——存在某个数$E$使得$\hat{E}\Psi=E\Psi$。不过,处于$\Psi$状态系统的能量是多少呢?两个能量本征态的叠加态又意味着什么呢?它的物理意义是什么?

首先我们能从拉格朗日量的$\mathcal{U}(1)$对称性中得到一点提示,它表明从$\Psi$的运动方程得到的解没有直接的物理意义\mpar{如果假定$\Psi$直接描述粒子,那么它的$\mathcal{U}(1)$变换$\Psi'=e^{i\alpha}\Psi$(也是运动方程的解)描述的是什么呢?}。

另一方面,注意到运动方程所有的解都是$\vec{x},t$的函数\mpar{下一节会对这点详细说明。},即$\Psi=\Psi(\vec{x},t)$。

标准的诠释是,{\bf 波函数}$\Psi(\vec{x},t)$模长的平方$|\Psi(\vec{x},t)|^2$表示粒子位置的概率密度。注意$\mathcal{U}(1)$变换对于$|\Psi|^2=\Psi^\dagger\Psi\to(\Psi')^\dagger(\Psi')=\Psi^\dagger e^{-i\alpha}e^{i\alpha}\Psi=\Psi^\dagger\Psi$这样的量是没有影响的。换句话说(以一维情形为例),$\Psi(x,t)$是在$[x, x+dx]$区间内测量到粒子的概率密度\footnote{即在区间$[x, x + dx]$测量到粒子的概率为$|\Psi(x, t)|^2 dx$.}。因此,如果对整个空间积分的话,必定有
\begin{align}
\label{equ8.10}
	\int dx\Psi^*(x,t)\Psi(x,t)\overset{!}{=}1
\end{align}
这称为波函数的归一化,因为在全空间中找到一个粒子的概率必定为$100\%=1$。

如果要预言其他物理量,则必须把波函数用相应的本征函数基展开。比如说用能量本征基展开:$\Psi=c_1\Phi_{E_1}+c_2\Phi_{E_2}+\cdots$。标准的量子力学诠释是,测量$\Psi$状态的体系的能量,所得结果只能为能量本征值,测量结果为本征值$E_1$的概率为$\Psi$与能量本征函数$\Phi_{E_1}$的“交叠”(的绝对值平方):
\[P(E_1)=\left|\int dx\Phi^*_{E_1}(x,t)\Psi(x,t)\right|^2 \]
在上面的例子中,它就是
\begin{align}
\label{equ8.11}
	\begin{split}
	P(E_1)&=\left|\int dx\Phi^*_{E_1}(x,t)\Psi(x,t)\right|^2 =\left|\int dx\Phi^*_{E_1}(x,t)(c_1\Phi_{E_1}+c_2\Phi_{E_2}+\cdots)\right|^2 \\
	&=\left|c_1\underbrace{\int dx\Phi^*_{E_1}(x,t)\Phi_{E_1}}_{=1\text{ 如之前所讲}}+\underbrace{\int dx\Phi^*_{E_1}(x,t) (c_2\Phi_{E_2}+\cdots)}_{\mathclap{=0\text{ 因为本征态是正交的}}}\right|^2\\
	&=|c_1|^2
	\end{split}
\end{align}
类似的,可以将$\Psi(\vec{x}, t)$用动量本征函数展开:
\[\Psi(\vec{x},t)=\frac{1}{\sqrt{2\pi}}\int_{-\infty}^{\infty}d^3 p\Psi(\vec{p},t)e^{-i\vec{p}\cdot\vec{x}}, \]
则$\Psi(\vec{p},t)$是系统动量处于$[p,p+dp]$区间内的概率幅。

这种解释可以用来对系统做几率性预测,比如用下一节介绍的统计期望值的办法。之后我们会导出非相对论性量子力学的运动方程,并看两个例子。

\subsection[期望值]{Expectation Value \quad 期望值}\label{sec8.3.1}

统计学中,期望值是类比于加权平均定义的。比如,如果掷骰子十次得到的结果为$2,4,1,3,3,6,3,1,4,5$,那么它的平均值就是
\[\langle x \rangle = (2+4+1+3+3+6+3+1+4+5)\cdot\frac{1}{10}=3.2 \]
而另一种计算这个的方法是计算每种相同结果的权重,给出经验几率
\[\langle x \rangle = \frac{2}{10}\cdot1+\frac{1}{10}\cdot2+\frac{3}{10}\cdot3+\frac{2}{10}\cdot4+\frac{1}{10}\cdot5+\frac{1}{10}\cdot6=3.2 \]
一般的,我们可以写
\begin{align}
\label{equ8.12}
	\langle x\rangle = \sum_i\rho_i x_i
\end{align}
其中$\rho_i$表示几率。对一个连续分布的情况同样有
\begin{align}
\label{equ8.13}
	\langle x\rangle = \int dx\rho(x)x
\end{align}

类比上面的形式,在量子力学中物理量$\hat{\mathcal{O}}$期望值的定义为
\begin{align}
\label{equ8.14}
	\langle \hat{\mathcal{O}}\rangle = \int d^3x\Psi^*\hat{\mathcal{O}}\Psi
\end{align}
一般的,要把$\Psi$按照$\hat{\mathcal{O}}$的本征函数展开,比如说动量本征函数。然后,把算符$\hat{\mathcal{O}}$作用在这些态上,得到对应的本征值,这样就得到了加权和。

举个例子,处于$\Psi$态的粒子的坐标期望值为
\begin{align}
\label{equ1.15}
	\langle \hat{x}\rangle = \int d^3x\Psi^*\hat{x}\Psi= \int d^3x\Psi^*{x}\Psi= \int d^3x\, x\underbrace{\Psi^*\Psi}_{\mathclap{\text{在位置}x\text{的几率密度}}}
\end{align}

为了计算方便,下面取Klein-Gordon方程的非相对论极限。


\section[Schrödinger 方程]{Schrödinger Equation \quad Schrödinger 方程}\label{sec8.4}
Klein-Gordon方程有平面波解
\[\Phi = \mathrm{e}^{\pm ip_\mu x^\mu} \equiv \mathrm{e}^{\pm ip\cdot x} \]
其中,$p_\mu=(E,\vec{p})^T$,是粒子四动量(满足守恒律):
\begin{align}
	0&=(\partial_\mu\partial^\mu+m^2)\Phi \notag \\
	&=(\partial_\mu\partial^\mu+m^2) \mathrm{e}^{\pm ip_\mu x^\mu} \notag \\
	&=(i^2 p_\mu p^\mu+m^2) \mathrm{e}^{\pm ip_\mu x^\mu}=0 \notag \\
\label{equ8.16}
	&=(-m^2+m^2) \mathrm{e}^{\pm ip_\mu x^\mu}=0\quad \checkmark
\end{align}

我们可以把解写成稍不同的形式
\[\Phi = \mathrm{e}^{\pm ip_\mu x^\mu}=\Phi= \mathrm{e}^{i(-Et+\vec{x}\cdot\vec{p})} \]
$\Phi$的时间部分是$\mathrm{e}^{-iEt}$,即$\Phi\propto \mathrm{e}^{-iEt}$。从\eqref{equ8.2}式可知
\[E=\sqrt{\vec{p}^2+m^2}=\sqrt{m^2\left(\frac{\vec{p}^2}{m^2}+1\right)}=m\sqrt{\frac{\vec{p}^2}{m^2}+1}. \]
在非相对论极限$|\vec{p}|\ll m$下,即研究对象的移动速度比光速要慢很多,从而其动量比质量也要小很多的情形下,可以对能量进行Taylor展开:
\[\begin{split}
E&=m\left(1+\frac{1}{2}\frac{\vec{p}^2}{m^2}+\cdots\right)\\
&\to E\approx\underbrace{m}_{\mathclap{\text{静质量}}}+\underbrace{\frac{\vec{p}^2}{2m}}_{\mathclap{\text{动能}}}.
\end{split} \]
于是
\begin{align}
\label{equ8.17}
	\Phi=\mathrm{e}^{i(-Et+\vec{x}\cdot\vec{p})}\approx \mathrm{e}^{-imt}\underbrace{\mathrm{e}^{i\vec{p}\cdot\vec{x}-i(p^2/2m)t}}_{\equiv\phi(\vec{x},t)}=\mathrm{e}^{-imt}\phi(\vec{x},t)
\end{align}

由$|\vec{p}|\ll m$可知静质能远大于粒子动能,因此$\phi(\vec{x}, t)$的振动频率远低于$\mathrm{e}^{-imt}$. 把$\mathrm{e}^{-imt}\phi(\vec{x},t)$作为试探解\footnote{译者注:ansatz是个很奇怪的词,我不知道怎么翻译。。比如Bethe-Ansatz,叫Bethe近似嘛也不对,叫啥都不对。}带入Klein-Gordon方程可得\footnote{原文下式有误,译文已按照勘误表改正。}
\[(\partial_\mu\partial^\mu+m^2) \mathrm{e}^{-imt}\phi(\vec{x},t)=(\partial_0\partial^0 - \partial_i\partial^i+m^2) \mathrm{e}^{-imt}\phi(\vec{x},t)=0. \]
根据莱布尼兹律,$\partial_t \mathrm{e}^{-imt}(\dots)= \mathrm{e}^{-imt}(-im+\partial_t)(\dots)$,即简单的把微分算符作用出来,我们得到
\[\mathrm{e}^{-imt} \big( (-im + \partial_t)^2 + \partial_i\partial^i + m^2 \big) \phi(\vec{x},t) = 0, \]
两边可以除掉$\mathrm{e}^{-imt}$,因为它肯定不是零。因此,
\[\begin{split}
\to \big((-im+\partial_t)^2+\nabla^2+m^2 \big)\phi(\vec{x},t)=0\\
\to \big( -m^2 - 2im\partial_t + (\partial_t)^2 + \nabla^2 + m^2 \big) \phi(\vec{x},t)=0.
\end{split} \]
把第三项
\begin{align}
	(\partial_t)^2 \phi(\vec{x},t) &= (\partial_t)^2 \exp\big[i\vec{p}\cdot\vec{x}-i(\vec{p}^2/2m)t \big] \notag \\
\label{equ8.18}
	&= \left(\frac{\vec{p}^2}{2m}\right)^2 \exp\big[ i\vec{p}\cdot\vec{x} - i(\vec{p}^2/2m)t \big] \propto\frac{p^4}{m^2}.
\end{align}
与第二项对比
\begin{align}
	im\partial_t\phi(\vec{x},t) &= im\partial_t\exp \big[i\vec{p}\cdot\vec{x} - i(\vec{p}^2/2m)t \big] \notag \\
\label{equ8.19}
	&= m \left( \frac{\vec{p}^2}{2m} \right) \exp\big[i\vec{p}\cdot\vec{x}-i(\vec{p}^2/2m)t \big] \propto p^2.
\end{align}
这表明,在$|\vec{p}|\ll m$的极限下,我们可以忽略掉第三项,因此
\[\begin{split}
(-2im\partial_t+\nabla^2)\phi(\vec{x},t)=0\\
\underbrace{\to}_{\mathclap{\text{除掉}(-2m)}}\left(i\partial_t-\frac{1}{2m}\nabla^2\right)\phi(\vec{x},t)=0.
\end{split}\]



\subsection[有外场的 Schrödinger 方程]{Schrödinger Equation with External Field 有外场的 Schrödinger 方程}\label{sec8.4.1}

此外,我们可以对{\bf 相互作用}Klein-Gordon方程\eqref{eq7.43},即描述有质量的$0$自旋场与无质量的自旋$1$场(光子场)的方程,做同样的非相对论性近似。于是我们得到
\begin{align}
\left(i\partial_t-\frac{1}{2m}\left(\nabla-ia\vec{A}\right)^2+q\Phi\right)\phi(\vec{x},t)=0
\end{align}

\section[从波动方程到粒子的运动]{From Wave Equations to Particle Motion 从波动方程到粒子的运动}\label{sec8.5}
\subsection[例:自由粒子]{Example: Free Particle 例:自由粒子}\label{sec8.5.1}
\subsection[例:盒子中的粒子]{Example: Particle in a Box 例:盒子中的粒子}\label{sec8.5.2}
\subsection[Dirac符号]{Dirac Notation Dirac符号}\label{sec8.5.3}
\subsection[例:盒子中的粒子,续]{Example: Particle in a Box, Again 例:盒子中的粒子,续}\label{sec8.5.4}
\subsection[自旋]{Spin 自旋}\label{sec8.5.5}
\section[Heisenberg 不确定性原理]{Heisenberg’s Uncertainty PrincipleHeisenberg 不确定性原理}\label{sec8.6}

现在可以来讨论量子力学中最奇妙的特征之一了。从上节我们已经知道,对粒子x方向上自旋的测量会破坏我们已知的在z方向上粒子自旋的信息,这一情况在量子力学中的很多测量中都存在。我们可以从\mpar{我们把自旋算子和对应的有限维旋转生成元认同,其满足对易关系$[J_i,J_j]=J_i J_j -J_j J_i=i\epsilon_{ijk}J_k\ne 0 \to J_i J_j \ne J_j J_i$。 举个例子来说,如果我们描述的是自旋1/2的粒子,那么我们应采用对应的二维表示$J_i={\sigma_i \over 2}$。} $\hat S_x \hat S_z \ne \hat S_z \hat S_x$ 出发来考察这一点,此式说明先对z 方向自旋后对x 方向自旋进行测量和先对x 方向自旋后对z 方向自旋进行测量的结果是不同的。这一点并不令人惊奇,因为对z 方向自旋进行测量后,系统应处于$\hat S_z$ 的一个本征态上,而对x 方向自旋进行测量后,系统应处于$\hat S_x$ 的本征态上,而$\hat S_z$ 和$\hat S_x$的本征态均不同,最后的结果也就并不相同。
\par
我们可以换种角度来描述这个现象:{\bf 我们不能同时确定系统z方向和x方向的自旋状态!}每次我们测量z方向的自旋时,都会让x方向的自旋信息再次变为未知,反之亦然。这一结论对于z 方向/x方向和y方向的自旋也是一样的。
\par
你也许觉得自旋是比较奇怪的物理量,但我们在粒子位置和动量的测量上也发现了同样的结果。回顾一下式(5.3),为了方便我们在这里重新写一遍:
\begin{equation}
[\hat p_i,\hat x_j]=\hat p_i \hat x_j - \hat x_j \hat p_j =i \delta_{ij}
\end{equation}
由上我们看出,对特定方向的粒子位置进行测量,会导致之前对其该方向动量的测量结果再次变得不确定。换句话说,我们不能同时{\it 足够精确地}确定粒子在同一方向上的位置与动量。注意,只有沿相同方向的位置和动量算子的对易子才不为零
\mpar{Kronecher delta函数$\delta_{ij}$在$i\ne j$时为0,$i=j$时为1,其定义在附录B.5.5。},所以y方向上动量的测量不会干扰我们对x方向上位置的测量。
\par
每次我们测量动量会使位置信息变得不确定,反之亦然,这就是著名的\textbf{Heisenberg 不确定原理}。同样的现象也发生在沿不同方向的角动量上,因为其对应的对易子也不为零。总而言之,我们可以检查任意两个物理量的对易性,如果它们不对易,它们就不能同时{\it 足够精确地}被确定。
\par
这一点可能并不像看上去那么令人惊奇。量子力学使用对应对称的生成元作为测量算子,比如说,对动量的测量等价于平移生成元的作用\mpar{记住,系统的平移不变性使我们导出动量守恒定律。},这一生成元将我们的系统平移了一小点,所以粒子的位置也被改变了。真正令人惊奇的是大自然就是这么运作的,许多年来有很多实验验证Heisenberg不确定原理,它们都证明了这个原理的正确性。



\section[对几种诠释的评议]{Comments on Interpretations \quad 对几种诠释的评议}\label{sec8.7}

\section*{阅读建议}

\section[附录:Dirac旋量分量的诠释]{Appendix: Interpretation of the Dirac Spinor Components \quad 附录:Dirac旋量分量的诠释}附录:\label{sec8.8}

\section[附录:解Dirac 方程]{Appendix: Solving the Dirac Equation 附录:解Dirac 方程}\label{sec8.9}

{\it 和上节一样,我们把Dirac自旋量中的二分量量记做{\rm u}和{\rm v},把四分量量记做$u$和$v$。也就是说例如$u_1,u_2$的记号代表的是四分量量,在不特别区分这些量的时候我们只写$u$。同时${\rm u_1}$和${\rm u_2}$是这个四分量量中的两个二分量量:$u=\pmm \rm u_1 \\ \rm u_2 \pmme $}。
\par
本附录中我们将在{\it 手性基底}上解{\it 静止参考系}下的Dirac方程。其他任意参考系的解能够通过本节解做伪转动变换得到。除了我们在上一节的讨论,我们在第六章讨论量子场论的时候也会用到这些解。\par
Dirac方程为:
\begin{equation}
(i\partial_\mu \gamma^\mu -m)\psi=0
\end{equation}
我们考虑其平面波解,设$\Psi=ue^{-ipx}$,其中$u$是一四分量量,因为矩阵$\gamma_\mu$是$4 \times 4$的。在静止参考系中,三维动量$\vec{p}=0$,指数项因此变为$-ipx=-i(p_0x_0-\vec{p}\vec{x})=-ip_0x_0$。现在使用我们本章初推导得到的相对论能量动量关系$E=\sqrt{|\vec{p}|^2+m^2}$,已知$p_0=E$和$x_0=t$,我们得到$-ipx=-iEt=-i\sqrt{|\vec{p}|^2+m^2}t=-imt$。把此式带入到Dirac方程中:
\begin{equation}
\begin{split}
&(i\partial_\mu\gamma^\mu-m)ue^{-imt}=0\\
&\to (i(\partial_0\gamma^0-\partial_i\gamma_i)-m)ue^{-imt}=0\\
&\to (i(-im)\gamma^0-m)ue^{-imt}=0\\
&\to (m\gamma^0-m)u=0\\
&\to (\pmm 0 & 1 \\ 1 & 0 \pmme -\pmm 1 & 0 \\ 0 & 1 \pmme)u=0\\
&\to \pmm -1 & 1 \\ 1 & -1 \pmme \pmm \rm u_1 \\ \rm u_2 \pmme =0\\
&\to \pmm \rm -u_1+u_2 \\ \rm u_1-u_2 \pmme =0
\end{split}
\end{equation}
注意这里矩阵中的1实际上代表的是$2 \times 2$的单位矩阵,所以$\rm u_1$和$\rm u_2$是二分量量。我们看出我们假设的平面波解在$\rm u_1 =u_2$的条件下符合方程,由此我们找到了Dirac方程两个线性独立的解:
\begin{equation}
\Psi_1=\pmm 1\\0\\1\\0 \pmme e^{-imt},~~~\Psi_2=\pmm 0\\1\\0\\1 \pmme e^{-imt}
\end{equation}
我们可以通过设$\tilde \Psi =ve^{ipx}$来得到另外两个解,在静止参考系下这一解形式化为$\tilde \Psi=ve^{imt}$,带入方程得
\begin{equation}
\begin{split}
&(i\partial_\mu\gamma^\mu-m)ve^{imt}=0\\
&\to (-m\gamma^0-m)v=0\\
&\to \pmm -1 & -1 \\ -1 & -1 \pmme \pmm \rm v_1 \\ \rm v_2 \pmme =0\\
&\to \pmm \rm -v_1-v_2 \\ \rm -v_1-v_2 \pmme =0
\end{split}
\end{equation}
我们由此得到时间依赖为$e^{imt}$的两个解,其满足$\rm -v_1=v_2$的条件。此时两个线性独立的解为:
\begin{equation}
\tilde \Psi_1=\pmm 1\\0\\-1\\0 \pmme e^{imt},~~~\tilde \Psi_2=\pmm 0\\1\\0\\-1 \pmme e^{imt}
\end{equation}



\section[附录:Dirac 旋量的不同的基]{Appendix: Dirac Spinors in Different Bases 附录:Dirac 旋量的不同的基}\label{sec8.10}

在拉格朗日量中,Dirac 旋量$\psi$总是和矩阵$\gamma_\mu$组合出现。
这个特性可以通过转换到一个不同的基来简化计算。
这种简化之所以行得通,是因为对于任意的可逆矩阵$N$,我们都可以通过在$\psi$和$\gamma_\mu$之间插入形如$1=N^{-1}N$的项来重新定义两者。
\begin{align}
  \partial_\mu\bar{\psi}\gamma_\mu\psi = \partial_\mu\bar{\psi}\underbrace{N^{-1}N}_{\mathclap{=1}}\gamma_\mu\underbrace{N^{-1}N}_{\mathclap{=1}}\psi = \partial_\mu\underbrace{\bar{\psi}N^{-1}}_{\mathclap{\equiv\bar{\psi}'}}\underbrace{N\gamma_\mu N^{-1}}_{\mathclap{\equiv\gamma'_\mu}}\underbrace{N\psi}_{\mathclap{\equiv\psi'}}
\end{align}
目前为止我们在本书中使用的基被称为手性基或者 Weyl 基。
而习惯上 Dirac 方程的求解中用的是另一种基,被称为质量基或者 Dirac 基。
在我们目前为止使用的手性或 Weyl 基中,Dirac 拉格朗日量
\begin{align}
  \mathcal{L}_D=i\chi_L^\dagger\sigma^\mu\partial_\mu \chi_L+i\xi_R^\dagger\bar{\sigma}^\mu\partial_\mu \xi_R-m\chi_L^\dagger\xi_R-m\xi_R^\dagger\chi_L
\end{align}
具有非对角的质量项——也就是混合了不同的态的质量项。
我们可以利用选择基的自由度来找到一组使得质量项对角的基,并将之称为质量基。

这意味着我们想要形如$\psi^\dagger m\psi$的质量项,其中$m=\begin{pmatrix}m_1&0\\0&m_2\end{pmatrix}$,
于是我们就得到了质量项的形式是
\begin{align}
  \bar{\psi}'M'\psi'=(\psi')^\dagger%此处原书中不带',应为印刷错误,但是勘误表中没有。带‘应为正确形式,故改正并说明。
  \gamma'_0M'\psi'=\begin{pmatrix}u'\\v'\end{pmatrix}^\dagger\begin{pmatrix}m_1&0\\0&m_2\end{pmatrix}
  \begin{pmatrix}u'\\v'\end{pmatrix}=(u')^\dagger m_1u'+(v')^\dagger m_2v',
\end{align}
而我们正在处理的是
\begin{align}
  \bar{\psi}M\psi=\psi^\dagger\gamma_0M\psi=\begin{pmatrix}\chi_L\\\xi_R\end{pmatrix}^\dagger\begin{pmatrix}0&m\\m&0\end{pmatrix}
  \begin{pmatrix}\chi_L\\\xi_R\end{pmatrix}=m\chi_L^\dagger\xi_R+m\xi_R^\dagger\chi_L.
\end{align}
我们目前为止一直使用的后一个基很容易体现出手性,然而质量基下的 Dirac 旋量会更容易和物理中的传播粒子建立联系。

为了得到第二个形式和第一个形式间的关系,我们需要把矩阵$M=\begin{pmatrix}0&m\\m&0\end{pmatrix}=m\begin{pmatrix}0&1\\1&0\end{pmatrix}$对角化。
这个矩阵的对角化可以通过矩阵$N=\frac{1}{\sqrt{2}}\begin{pmatrix}-1&1\\1&1\end{pmatrix}$来实现,
也就是说
\begin{align}
  N^{-1}\underbrace{\begin{pmatrix}-m&0\\0&m\end{pmatrix}}_{\mathclap{\equiv M'}}N=M
\end{align}
\begin{align}
  \rightarrow m\frac{1}{\sqrt{2}}\begin{pmatrix}-1&1\\1&1\end{pmatrix}^-1\begin{pmatrix}-1&0\\0&1\end{pmatrix}\frac{1}{\sqrt{2}}
  \begin{pmatrix}-1&1\\1&1\end{pmatrix}=m\begin{pmatrix}0&1\\1&0\end{pmatrix}
\end{align}
于是我们就可以相应地重新定义 Dirac 旋量
\begin{align}
\begin{split}
  \bar{\psi}M\psi=&\bar{\psi}\underbrace{NN^{-1}}_{\mathclap{=-1}}M\underbrace{NN^{-1}}_{\mathclap{=-1}}\psi\\
  =&\underbrace{\bar{\psi}N}_{\mathclap{\equiv\bar{\psi}'}}
  \underbrace{N^{-1}MN}_{\mathclap{\equiv M'}}\underbrace{N^{-1}\psi}_{\mathclap{\equiv\psi'}}\\
  =&\bar{\psi}'M'\psi'
  \end{split}
\end{align}

观察一下在这个基下的手性投影算符$P_L=\frac{1-\gamma_5}{2}$是很有帮助的。
因此我们需要得出相应的$\gamma_5$矩阵
\begin{align}
\begin{split}
  \bar{\gamma}_5=N^{-1}\gamma_5N=&\frac{1}{\sqrt{2}}\begin{pmatrix}1&1\\1&-1\end{pmatrix}\begin{pmatrix}1&0\\0&-1\end{pmatrix}
  \frac{1}{\sqrt{2}}\begin{pmatrix}1&1\\1&-1\end{pmatrix}\\
  =&\frac{1}{2}\begin{pmatrix}1&1\\1&-1\end{pmatrix}\begin{pmatrix}1&1\\-1&1\end{pmatrix}\\
  =&\begin{pmatrix}0&1\\1&0\end{pmatrix}
\end{split}
\end{align}
与此对应的本征向量是$\frac{1}{\sqrt{2}}\begin{pmatrix}1\\-1\end{pmatrix}$和$\frac{1}{\sqrt{2}}\begin{pmatrix}1\\1\end{pmatrix}$。
这表示手性本征态现在由具有上下分量的 Dirac 旋量描述,
例如,左手态在这个基下的形式就是$\frac{1}{\sqrt{2}}\begin{pmatrix}1\\-1\end{pmatrix}$
相比之下,在手性基下$\gamma_5$是对角的,左手本征态由只有上分量的 Dirac 旋量$\psi_L=\begin{pmatrix}\chi_L\\0\end{pmatrix}$给出,
右手本征态则由只有下分量的 Dirac 旋量$\psi_R=\begin{pmatrix}0\\\xi_R\end{pmatrix}$给出。

在这个基下手性投影算符写成
\begin{align}
  P_L=\frac{1-\gamma_5}{2}=\frac{1}{2}\begin{pmatrix}1&-1\\-1&1\end{pmatrix}
\end{align}

\subsection[质量基下 Dirac 方程的解]{Solutions of the Dirac Equation in the Mass Basis 质量基下 Dirac 方程的解}\label{sec8.10.1}

我们可以求解质量基下的 Dirac 方程
\begin{align}
  (i\gamma_\mu\partial^\mu-m)\Psi=0
\end{align}
通过假设$\psi=ue^{-ipx}$,可以导出
\[(\gamma_\mu p^\mu-m)ue^{-ipx}=0\]
\[\rightarrow(\gamma_\mu p^\mu-m)u=0.\]
等价地,我们可以假设$\psi=ve^{-ipx}$导出
\[(-\gamma_\mu p^\mu-m)ve^{ipx}=0\]
\[\rightarrow(-\gamma_\mu p^\mu-m)v=0.\]
类似在手性基中的解,我们这里也在静止坐标系中求解,也就是$\vec{p}=0$。
我们做这样的选择之所以可行,是因为物理规律在所有参照系中都是相同的,所以我们可以选择最符合我们需求的一个。
在这个参照系中,由于$p_i=0$,所以我们有
\[\rightarrow(\gamma_0p^0-m)u=0\]
\[\rightarrow(-\gamma_0p^0-m)v=0.\]
此外还有$p_0=E$,并且我们可以使用我们在本章开头(式\ref{eq:8.2})推导的相对论能-动量关系。
在静止参照系中我们有$E=\sqrt{(p_i)^2+m^2}=m$。
现在我们可以利用上面给出的矩阵$N$和$\gamma'_0=N^{-1}\gamma_0N$计算出$\gamma_0$在质量基或 Dirac 基中的精确形式。
要记得在$m$中我们还包含着一个单位矩阵,因此
\[\rightarrow\left(\begin{pmatrix}1&0\\0&-1\end{pmatrix}m-m\begin{pmatrix}1&0\\0&1\end{pmatrix}\right)u=0\]
\[\rightarrow\left(-\begin{pmatrix}1&0\\0&-1\end{pmatrix}m-m\begin{pmatrix}1&0\\0&1\end{pmatrix}\right)v=0.\]
\[\rightarrow\begin{pmatrix}0&0\\0&-2\end{pmatrix}u=0\]
\[\rightarrow\begin{pmatrix}-2&0\\0&0\end{pmatrix}v=0.\]
要记得每个 Dirac 旋量都包含两个二分量的变量,我们可以推出$u$的下面的二分量变量和$v$的上面的二分量变量必定是0:
\[\rightarrow\begin{pmatrix}0&0\\0&-2\end{pmatrix}\begin{pmatrix}u_1\\u_2\end{pmatrix}=\begin{pmatrix}0\\-2u_2\end{pmatrix}=0\rightarrow u_2=0\]
\[\rightarrow\begin{pmatrix}-2&0\\0&0\end{pmatrix}\begin{pmatrix}v_1\\v_2\end{pmatrix}=\begin{pmatrix}-2v_1\\0\end{pmatrix}=0\rightarrow v_1=0.\]

我们可以看到在这个基下物理的传播粒子(也就是 Dirac 方程的解)可以用只含有上分量的旋量描述,而等价地对于反粒子则只含有下分量。
从而我们可以再次得到四个线性独立的解
\begin{align}
\Psi'_1=\begin{pmatrix}1\\0\\0\\0\end{pmatrix}e^{-imt}~~~~~~\Psi'_2=\begin{pmatrix}0\\1\\0\\0\end{pmatrix}e^{-imt}
\end{align}
和

\begin{align}
\Psi'_3=\begin{pmatrix}0\\0\\1\\0\end{pmatrix}e^{imt}~~~~~~\Psi'_4=\begin{pmatrix}0\\0\\0\\1\end{pmatrix}e^{imt}
\end{align}

在这个参照系中这个基下的一般解,可以写成线性组合的形式
\begin{align}
\psi=ue^{-ipx}+ve^{ipx}=\begin{pmatrix}u_1\\0\end{pmatrix}e^{-ipx}+\begin{pmatrix}0\\v_1\end{pmatrix}e^{ipx}
\end{align}
并且我们可以通过将这个解做一个 Lorentz 推动变换得到任意一个参照系中的解。
此外,最一般的解是所有动量和自旋态\mpar{注意 Weyl 旋量的两个分量代表不同的自旋态。}的解的叠加

\begin{align}
\Psi=\sum_r\sqrt{\frac{m}{(2\pi)^3}}\int\frac{d^3p}{\sqrt{E_p}}\left(c_r(p)u_r(p)e^{-ipx}+d^\dagger_r(p)v_r(p)e^{+ipx}\right)
\end{align}
