%!TEX encoding = UTF-8 Unicode

%----------------------------------------------------------------------------------------
%	CHAPTER 6
%----------------------------------------------------------------------------------------

\chapterimage{chapter_head_1.pdf} % Chapter heading image

\chapter[自由场理论]{Free Theory 自由场理论}\label{chap6}

在这章中,我们建立自由的体系的物理理论,也就是说无相互作用的情况下的对称性给出的场\mpar{尽管我们这里针对的是场,后面会看到这里得到的方程同样可以用来处理粒子}。我们会
\begin{itemize}
\item 利用Lorentz群的$(0,0)$表示给出Klein-Gordon方程
\item 利用Lorentz群的$(\tfrac{1}{2},0)\oplus(0,\tfrac{1}{2})$表示给出Dirac方程
\item 利用Lorentz群的矢量$(\tfrac{1}{2},\tfrac{1}{2})$表示给出Proca方程,在无质量极限下退化到著名的Maxwell方程组
\end{itemize}

\section[Lorentz协变性与不变量]{Lorentz Covariance and Invariance Lorentz协变性与不变量}\label{sec6.1}

在下面的章节中,我们会得到粒子物理的标准模型- 这是我们拥有的最好的物理理论 -中的运动方程。我们期待这些房产在所有的惯性系中看起来都一样,因为如果不这样的话,我们就会对每一个可能的参考系都有各自的方程。狭义相对论指出并没有一个特别的参考系,所以这一点是没有意义的。这个术语被称为{\bf Lorentz协变性}。一个Lorentz协变的物体是说它按照某一种给定的Lorentz群来进行变换。比如说,矢量$A_\mu$,通过$\left(\frac{1}{2},\frac{1}{2}\right)$表示来进行变换,因此是Lorentz协变的;这其实是说,$A_\mu\to A'_{\mu'}$,但实际上这两个指代同一个东西,并不是完全不同的。另一方面,比如说,$A_1+A_3$就不是Lorentz协变的,因为它不按照Lorentz群的某种表示来变换。但这并不是说我们不知道它是如何变换的,因为它的变换规律从$A_\mu$的变换规律里面可以简单地得出,但是在不同的惯性系中它看起来完全不一样。在一个``推动''的参考系中它看起来或许像$A_2+A_4$。只涉及到Lorentz协变的东西的方程被称为Lorentz协变方程,比如
\[A_\mu+7B_\mu+C_\nu A^\nu D_\mu = 0 \]
就是一个Lorentz协变方程,因为在另一个参考系中,它看起来就是
\[A_\mu+7B_\mu+\underbrace{C_\nu A^\nu}_{\mathclap{=\text{一个Lorentz标量,根据$(0,0)$表示来进行变换}}} D_\mu = A'_\mu+7B'_\mu+C_\nu A^\nu D'_\mu = 0 \]
我们看到它看起来是一样的。一个只有部分是这样的东西的方程,一般来说并不是Lorentz协变的,从而在另一个惯性系里面就会不一样。

为了确定我们只使用Lorentz{\bf 协变}的方程,我们需要要求作用量$S$是Lorentz{\bf 不变}的。这是说它只能包含那些在参考系变换下保持不变的成分。换句话说,作用量里面只可以包含Lorentz变换下不变的部分。从作用量$S$中我们可以得到运动方程\mpar{回忆一下我们保持作用量最小以及得来的欧拉-拉格朗日方程,就是对系统的运动方程}。如果现在$S$依赖于参考系,也就是说得到的运动方程不再是Lorentz协变的了。

就像我们之前一章讨论的那样,我们可以用更严格的约束条件来要求拉格朗日量应该是不变的,因为这样的话作用量也就肯定是不变的了。

\section[Kelin-Gordon 方程]{Kelin-Gordon Equation Kelin-Gordon 方程}\label{sec6.2}

我们现在开始考虑最简单的情况:标量,其按照Lorentz群的$(0,0)$表示变化。我们需要找一个对应的拉格朗日量来确定标量的运动方程。一个符合我们限制的一般的拉格朗日量第\mpar{在\ref{sec4.2}节中有所讨论:我们只考虑$0, 1$和$2$阶的$\Phi$。只考虑最低可能的导数项的原因在后面就会说明。}是:

\begin{align}
\mathcal{L} = A\Phi^0+V\Phi+C\Phi^2+D\partial_\mu\Phi+E\partial_\mu\Phi\partial^\mu\Phi+F\Phi\partial_\mu\Phi
\end{align}
首先,需要注意我们考虑的是拉格朗日量密度$\mathcal{L}$而不是$L$本身,然后我们的物理理论可以从作用量
\begin{align}
S = \int dx\mathcal{L}
\end{align}
得出,其中$dx$可以理解为对时间和空间的积分。因此,诸如$\Phi\partial_\mu\partial^\mu\Phi$的项实际是没必要的,因为它等效于$\partial_\mu\Phi\partial^\mu\Phi$,从分部积分就可以直接地看到这一点\mpar{很直接的,边界项会被小区,因为在远处场很小。值得注意的是,这是由于我们物理里面的速度是有上限的(第\ref{sec2.3}节)。因此,很远处的场对于有限远的$x$位置的场不会有影响}。

不仅如此,Lorentz{\bf 不变形}要求拉格朗日量必须是一个标量。因此,所有奇数阶的$\partial_\mu$,比如$\partial_\mu\Phi$是禁止的。值得一提的是,如果常数,比如$a, c$,有一个Lorentz指标的话,这说明这些常数是一个$4$-矢量,标定了时空的方向从二破坏了空间的各向同性的要求。我们实际上可以忽略常数项,也就是$A=0$,因为我们的物理理论需要从欧拉-拉格朗日方程中得到,因此一个常数项对运动方程没有影响\mpar{考虑\eqref{eq4.10}:$\frac{\partial{\mathcal{L}}}{\partial\Phi}-\partial_\mu\left(\frac{\partial{\mathcal{L}}}{\partial(\partial_\mu\Phi)}\right)$,因此$\mathcal{L}\to\mathcal{L}+A$中的常数项$A$并不改变任何东西:$\frac{\partial({\mathcal{L}}+A)}{\partial\Phi}-\partial_\mu\left(\frac{\partial({\mathcal{L}}+A)}{\partial(\partial_\mu\Phi)}\right) = \frac{\partial{\mathcal{L}}}{\partial\Phi}-\partial_\mu\left(\frac{\partial{\mathcal{L}}}{\partial(\partial_\mu\Phi)}\right)$}。除此之外,我们可以忽略掉$\Phi$的线性项,也就是说$B=0$,因为从欧拉-拉格朗日方程中可以看出这一项只会给运动方程增加一个常数\mpar{$\mathcal{L}\to\mathcal{L}+B\Phi$使得$\frac{\partial(\mathcal{L}+B\Phi)}{\partial\Phi}-\partial_\mu\left(\frac{\partial(\mathcal{L}+B\Phi)}{\partial(\partial_\mu\Phi)}\right)=\frac{\partial\mathcal{L}}{\partial\Phi}-\partial_\mu\left(\frac{\partial{\mathcal{L}}}{\partial(\partial_\mu\Phi)}\right)+B=0$}。那么,剩下的项就是:
\begin{align}
\mathcal{L}=C\Phi^2+E\partial_\mu\Phi\partial^\mu\Phi
\end{align}

我们在本章开始的时候就指明了我们想建立一个自由场的理论,这就是说我们只有一个$\Phi$而不会有形如$\Phi_1\Phi_2$这种相互作用形式;我们在下一章会建立一种理论来描述它。

还有一件事情需要注意:我们最后留了两个常数$C$和$E$,用变分法我们可以把它俩最重收为一个常数,因为拉格朗日量整体差一个常数对物理没有影响\mpar{$\mathcal{L}\to C\mathcal{L}$得到
\[\begin{split}
\frac{\partial(C\mathcal{L})}{\partial\Phi}&-\partial_\mu\left(\frac{\partial(C\mathcal{L})}{\partial(\partial_\mu\phi)}\right)=0\\
&\updownarrow\\
\frac{\partial(\mathcal{L})}{\partial\Phi}&-\partial_\mu\left(\frac{\partial(\mathcal{L})}{\partial(\partial_\mu\phi)}\right)=0
\end{split} \]}。不仅如此,通常我们都在拉格朗日量前增加一个$\frac{1}{2}$,然后称另外的常数位$-m^2$\mpar{后面我们就会明白的看到为什么这样称呼这个常数:它和拉格朗日量用来描述例子质量的形式是一样的。}。因此,我们最重得到了
\begin{align}
\mathcal{L}=\frac{1}{2}(\partial_\mu\Phi\partial^\mu\Phi-m^2\Phi^2)
\end{align}

如果我们现在用变分来处理的话,也就是说把拉格朗日量带入到欧拉-拉格朗日方程\eqref{eq:4.10}的话,我们就得到了运动方程
\begin{align}
\begin{split}
&\frac{\mathcal{L}}{\partial\Phi}-\partial_\mu\left(\frac{\partial(\mathcal{L})}{\partial(\partial_\mu\phi)}\right)=0\\
&\rightarrow\frac{\partial}{\partial\Phi}\left(\frac{1}{2}(\partial_\mu\Phi\partial^\mu\Phi-m^2\Phi^2)\right)-\partial_\mu\left(\frac{\partial}{\partial(\partial_\mu\Phi)}\left(\frac{1}{2}(\partial_\mu\Phi\partial^\mu\Phi-m^2\Phi^2)\right)\right)=0 \\
&\rightarrow(\partial_\mu\partial^\mu+m^2)\Phi=0
\end{split}
\end{align}

这就是著名的{\bf Kelin-Gordon方程},是描述自由的自旋$0$的场/粒子的正确的方程。

\subsection[复Kelin-Gordon场]{Complex Kelin-Gordon Field 复Kelin-Gordon场}\label{sec6.2.1}

对于自旋$0$的场来说,我们可以构造Lorentz协变的拉格朗日量而不是用标量场的复共轭。然而对弈自旋$\frac{1}{2}$场$\Psi$来说就不一样了,这会带来非常有趣的结果\mpar{完全不必惊讶:每个自旋$\frac{1}{2}$的粒子都有反粒子。用复数场实际上就是同时考虑两个场,我们接下来会说到。因此,我们强制的令Lorentz协变性同时在{\bf 两个}(紧密联系的)场上作要求,其通常就被解释为粒子和反粒子场。}。而且,我们甚至也可以像很多教科书一样研究等价的Lorentz协变的拉格朗日量
\[L=\partial_\mu\phi^\dagger\partial^\mu\phi-m^2\phi^\dagger\phi \]
这实际上与考虑{\bf 两个}同等质量标量场的相互作用是一样的:
\[L=\frac{1}{2}\partial_\mu\phi_1\partial^\mu\phi_1-\frac{1}{2}m^2\phi_1^2+\frac{1}{2}\partial_\mu\phi_2\partial^\mu\phi_2-\frac{1}{2}m^2\phi_2^2 \]
因为我们有
\[\phi\equiv\frac{1}{\sqrt2}(\phi_1+i\phi_2) \]
Lorentz对称性决定了拉格朗日量的形式。基本的标量(自旋$0$)粒子非常罕见,事实上,只有Higgs玻色子是实验上验证了的这种例子,但是这个拉格朗日量可以用来描述复合系统,比如介子。至此,我们就不在研究这种拉格朗日量了,而且大多数教科书中也仅仅是用它来做练习用。

\section[Dirac方程]{Dirac Equation Dirac 方程}\label{sec6.3}
\begin{quote}
Dirac的另一个故事是他第一次遇到Feynman的时候,在漫长的寂静之后他说:``我有一个方程。你也有吗?''\\
\phantom{Dirac的另一个故事是他第一次遇到}--{\bf Anthony Zee\mpar{Anthony Zee. {\it Quantum Field Theory in a Nutshell}. Princeton University Press, 1st edition, 3 2003. ISBN 9780691010199}}
\end{quote}

我们这节想要研究自由的自旋$\frac{1}{2}$场/粒子的运动方程。我们会使用Lorentz群的$(\frac{1}{2},0)\oplus(0,\frac{1}{2})$表示,因为要描述宇称变换下的对称性的理论必须同时有$(\frac{1}{2},0)$和$(0,\frac{1}{2})$表示\mpar{我们在第\ref{sec3.7.9}节中说明了,宇称变换会把$\frac{1}{2},0)$表示变成$(0,\frac{1}{2})$表示}。而在这种对称群表示下做变换的东西我们叫做Dirac旋量,同时包含右手旋量和左手旋量。就像已经在第\ref{sec3.7.9}节中讨论的一样,Dirac旋量定义为
\begin{align}
\Psi\equiv\left(\begin{matrix}\xi_L\\ \chi_R\end{matrix}\right)=\left(\begin{matrix}\xi_a\\ \chi^{\dot{a}}\end{matrix}\right)
\end{align}

现在,我们要找Lorentz协变的东西来建立Dirac旋量。


\section[Proca方程]{Proca Equation Proca 方程}\label{sec6.4}







