%!TEX encoding = UTF-8 Unicode

%----------------------------------------------------------------------------------------
%	CHAPTER 6
%       translator: laserdog
%       proofreader: Surgam Identidem
%----------------------------------------------------------------------------------------

\chapterimage{chapter_head_1.pdf} % Chapter heading image

\chapter[自由场理论]{Free Theory \quad 自由场理论}\label{chap6}

本章建立{\bf 自由}体系的物理理论,也就是说无相互作用的情况下的对称性给出的场\mpar{尽管这里针对的是场,后面会看到这里得到的方程同样可以用来处理粒子}。我们会
\begin{itemize}
\item 利用Lorentz群的$(0,0)$表示给出Klein-Gordon方程
\item 利用Lorentz群的$(\tfrac{1}{2},0)\oplus(0,\tfrac{1}{2})$表示给出Dirac方程
\item 利用Lorentz群的矢量$(\tfrac{1}{2},\tfrac{1}{2})$表示给出Proca方程,在无质量极限下退化到著名的Maxwell方程组
\end{itemize}

\section[Lorentz协变性与不变量]{Lorentz Covariance and Invariance \quad Lorentz协变性与不变量}\label{sec6.1}

本章会导出粒子物理的标准模型(这是我们拥有的最好的物理理论)中的运动方程。我们期待这些方程在所有的惯性系中都一样,因为如果不这样的话,每一个参考系都有各自的方程,而狭义相对论指出并没有一个特别的参考系,所以各惯性系中方程形式相同。这称为{\bf Lorentz协变性}。一个Lorentz协变的对象是说它按照某一种给定的Lorentz群的表示来进行变换。比如说,矢量$A_\mu$,通过$\left(\frac{1}{2},\frac{1}{2}\right)$表示变换,因此是Lorentz协变的;这其实是说,Lorentz变换:$A_\mu \to A'_\mu$,$A_\mu, A'_{\mu}$实际是同一个对象(而非完全不同的)。另一方面,比如说,$A_1+A_3$就不是Lorentz协变的,因为它不按照Lorentz群的某种表示来变换。但这并不是说我们不知道它是如何变换的,因为它的变换规律从$A_\mu$的变换规律里面可以简单地得出,但是在不同的惯性系中它看起来完全不一样。经推动变换到另一个惯性系之后它看起来或许像$A_2+A_4$。只包含Lorentz协变的对象的方程称为Lorentz协变方程,比如
\[A_\mu+7B_\mu+C_\nu A^\nu D_\mu = 0 \]
就是一个Lorentz协变方程,因为在另一个参考系中,它看起来就是
\[A_\mu+7B_\mu+\underbrace{C_\nu A^\nu}_{\mathclap{=\text{一个Lorentz标量,根据$(0,0)$表示来进行变换}}} D_\mu = A'_\mu+7B'_\mu+C_\nu A^\nu D'_\mu = 0 \]
它在不同惯性系看起来是一样的。只有一部分是这样的方程一般不是Lorentz协变的,它在另一个惯性系里面会不一样。

为了导出Lorentz{\bf 协变}的方程,我们要求作用量$S$是Lorentz{\bf 不变}的。这是说$S$只能包含在参考系变换下保持不变的成分。换句话说,作用量里面只可以包含Lorentz变换下不变的成分。从作用量$S$中我们可以得到运动方程\mpar{回忆一下,我们使作用量取极小值进而得到的欧拉-拉格朗日方程就是系统的运动方程。}。如果$S$依赖于参考系,那么运动方程就不是Lorentz协变的了。

就像前一章提到的,可以用更严格的约束条件 --- 要求拉格朗日量是Lorentz不变的,这样作用量也就肯定是不变的了。

\section[Klein-Gordon 方程]{Klein-Gordon Equation \quad Klein-Gordon 方程}\label{sec6.2}

下面从最简单的情况开始:标量,它按照Lorentz群的$(0,0)$表示变换。我们要找出对应的拉格朗日量来确定标量的运动方程。符合要求的拉格朗日量的一般形式为\mpar{在\ref{sec4.2}节中有所讨论:我们只考虑$0, 1$和$2$阶的$\Phi$。只考虑最低可能的导数项的原因在后面就会说明。}:

\begin{align}
\label{equ6.1}
\mathscr{L} = A\Phi^0+B\Phi+C\Phi^2+D\partial_\mu\Phi+E\partial_\mu\Phi\partial^\mu\Phi+F\Phi\partial_\mu\Phi
\end{align}
注意我们考虑的是拉格朗日量密度$\mathscr{L}$而不是$L$本身,物理理论从作用量
\begin{align}
\label{equ6.2}
S = \int d\mathbf{x} \mathscr{L}
\end{align}
得出,其中$d \mathbf{x}$可以理解为对时间和空间的积分。因此,诸如$\Phi\partial_\mu\partial^\mu\Phi$的项实际是没必要的,因为它等效于$\partial_\mu\Phi\partial^\mu\Phi$,从分部积分就可以直接看出\mpar{分部积分产生的边界项为零,因为在远处场很小。值得注意的是,这是由于我们物理里面的速度是有上限的(第\ref{sec2.3}节)。因此,很远处的场对于有限远的$x$位置的场不会有影响。}。

不仅如此,Lorentz{\bf 不变性}要求拉格朗日量必须是标量。因此,所有奇数阶的$\partial_\mu$,比如$\partial_\mu\Phi$是禁止的。值得一提的是,如果常数,比如$a, c$,有一个Lorentz指标的话,这说明这些常数是一个$4$-矢量,标定了时空的方向从而破坏了空间的各向同性。常数项实际上可忽略,也就是$A=0$,因为物理理论从欧拉-拉格朗日方程中导出,因此一个常数项对运动方程没有影响\mpar{考虑\eqref{equ4.10}式:$\frac{\partial{\mathscr{L}}}{\partial\Phi}-\partial_\mu\left(\frac{\partial{\mathscr{L}}}{\partial(\partial_\mu\Phi)}\right) = 0$,因此$\mathscr{L}\to\mathscr{L}+A$中的常数项$A$并不改变任何东西:$\frac{\partial({\mathscr{L}}+A)}{\partial\Phi}-\partial_\mu\left(\frac{\partial({\mathscr{L}}+A)}{\partial(\partial_\mu\Phi)}\right) = \frac{\partial{\mathscr{L}}}{\partial\Phi}-\partial_\mu\left(\frac{\partial{\mathscr{L}}}{\partial(\partial_\mu\Phi)}\right)$.}。除此之外,我们可以忽略掉$\Phi$的线性项,也就是说$B=0$,因为从欧拉-拉格朗日方程中可以看出这一项只会给运动方程增加一个常数\mpar{$\mathscr{L}\to\mathscr{L}+B\Phi$使得$\frac{\partial(\mathscr{L}+B\Phi)}{\partial\Phi}-\partial_\mu\left(\frac{\partial(\mathscr{L}+B\Phi)}{\partial(\partial_\mu\Phi)}\right)=\frac{\partial\mathscr{L}}{\partial\Phi}-\partial_\mu\left(\frac{\partial{\mathscr{L}}}{\partial(\partial_\mu\Phi)}\right)+B=0$,只产生运动方程中的可加常数。}。那么,剩下的项就是:
\begin{align}
\label{equ6.3}
\mathscr{L}=C\Phi^2+E\partial_\mu\Phi\partial^\mu\Phi
\end{align}

本章开始就指明了要建立一个自由场的理论,这意味着只有一个$\Phi$而不会有形如$\Phi_1\Phi_2$这种相互作用形式,这样的项在下一章 --- 描述相互作用的理论中考虑。

还有一件事情需要注意:最后留下了两个常数$C$和$E$,利用变分法可以把它俩最终收为一个常数,因为拉格朗日量整体差一个常数因子并无影响\mpar{$\mathscr{L}\to C\mathscr{L}$得到
\[\begin{split}
\frac{\partial(C\mathscr{L})}{\partial\Phi}&-\partial_\mu\left(\frac{\partial(C\mathscr{L})}{\partial(\partial_\mu\phi)}\right)=0\\
&\updownarrow\\
\frac{\partial(\mathscr{L})}{\partial\Phi}&-\partial_\mu\left(\frac{\partial(\mathscr{L})}{\partial(\partial_\mu\phi)}\right)=0
\end{split} \]}。不仅如此,通常我们都在拉格朗日量前增加一个$\frac{1}{2}$,然后称另外的常数为$-m^2$。\mpar{后面我们就会明白的看到为什么这样称呼这个常数:它正是拉格朗日量中的粒子质量。}因此,我们最终得到了
\begin{align}
\label{equ6.4}
\mathscr{L}=\frac{1}{2}(\partial_\mu\Phi\partial^\mu\Phi-m^2\Phi^2)
\end{align}

现在利用变分法,也就是说把拉格朗日量带入到欧拉-拉格朗日方程\eqref{equ4.10}式,我们就得到了运动方程
\begin{align}
\label{equ6.5}
\begin{split}
&\frac{\mathscr{L}}{\partial\Phi}-\partial_\mu\left(\frac{\partial(\mathscr{L})}{\partial(\partial_\mu\phi)}\right)=0\\
&\rightarrow\frac{\partial}{\partial\Phi}\left(\frac{1}{2}(\partial_\mu\Phi\partial^\mu\Phi-m^2\Phi^2)\right)-\partial_\mu\left(\frac{\partial}{\partial(\partial_\mu\Phi)}\left(\frac{1}{2}(\partial_\mu\Phi\partial^\mu\Phi-m^2\Phi^2)\right)\right)=0 \\
&\rightarrow(\partial_\mu\partial^\mu+m^2)\Phi=0
\end{split}
\end{align}

这就是著名的{\bf Klein-Gordon方程},是描述自旋$0$的自由场/粒子的方程。

\subsection[复Klein-Gordon场]{Complex Klein-Gordon Field \quad 复Kelin-Gordon场}\label{sec6.2.1}

对于自旋$0$的场可以构造Lorentz协变的拉格朗日量,而不是用标量场的复共轭。然而对于自旋$\frac{1}{2}$场$\Psi$来说就不一样了,这会带来非常有趣的结果\mpar{完全不必惊讶:每个自旋$\frac{1}{2}$的粒子都有反粒子。用复数场实际上就是同时考虑两个场,接下来会说到。因此,我们强制的令Lorentz协变性同时在{\bf 两个}(紧密联系的)场上作要求,这通常就解释为粒子和反粒子场。}。而且,我们甚至也可以像很多教科书一样研究等价的Lorentz协变的拉格朗日量
\[L=\partial_\mu\phi^\dagger\partial^\mu\phi-m^2\phi^\dagger\phi \]
这实际上与考虑{\bf 两个}等质量标量场的相互作用是一样的:
\[L=\frac{1}{2}\partial_\mu\phi_1\partial^\mu\phi_1-\frac{1}{2}m^2\phi_1^2+\frac{1}{2}\partial_\mu\phi_2\partial^\mu\phi_2-\frac{1}{2}m^2\phi_2^2 \]
因为我们有
\[\phi\equiv\frac{1}{\sqrt2}(\phi_1+i\phi_2) \]
Lorentz对称性决定了拉格朗日量的形式。基本的标量(自旋$0$)粒子非常罕见,事实上,只有Higgs玻色子是实验上验证了的,但是这个拉格朗日量可以描述复合系统,比如介子。因此,下面不再研究这种拉格朗日量了,而且大多数教科书中也仅仅是用它来作为习题。

\section[Dirac方程]{Dirac Equation \quad Dirac 方程}\label{sec6.3}
\begin{quote}
Dirac的另一个故事是他第一次遇到Feynman的时候,在漫长的寂静之后他说:“我有一个方程。你也有吗?”\\
\phantom{Dirac的另一个故事是他第一次遇到}--{\bf Anthony Zee\mpar{Anthony Zee. {\it Quantum Field Theory in a Nutshell}. Princeton University Press, 1st edition, 3 2003. ISBN 9780691010199}}
\end{quote}

我们这节想要研究自由的自旋$\frac{1}{2}$场/粒子的运动方程。我们会使用Lorentz群的$(\frac{1}{2},0)\oplus(0,\frac{1}{2})$表示,因为要描述宇称变换下的对称性的理论必须同时有$(\frac{1}{2},0)$和$(0,\frac{1}{2})$表示\mpar{我们在第\ref{sec3.7.9}节中说明了,宇称变换会把$(\frac{1}{2},0)$表示变成$(0,\frac{1}{2})$表示}。而在这种对称群表示下做变换的对象称为Dirac旋量,包括右手旋量和左手旋量。\ref{sec3.7.9}节讲过,Dirac旋量定义为
\begin{align}
\label{equ6.6}
\Psi\equiv\left(\begin{matrix}\chi_L\\ \xi_R\end{matrix}\right)=\left(\begin{matrix}\chi_a\\ \xi^{\dot{a}}\end{matrix}\right)
\end{align}

现在需要用Dirac旋量构造某个Lorentz不变量,从而放到拉格朗日量里面。第一步是从左手旋量和右手旋量出发构造不变量。

利用在第\ref{sec3.7.7}节介绍的Van-der-Waerden记号。两种可能的形式分别是\mpar{这里实际上有另外两种可能,称为Majorana质量项。我们已经知道如何使用自旋“度规”$\epsilon$来把自旋指标升或者降。我们可以写一个Lorentz不变的项,诸如$\epsilon(\xi_L)^\dagger\xi_L$,因为$\epsilon(\xi_L)^\dagger=\xi^a$的形式,有一个上标的右手旋量;但是这样写的话就会缺少自由度。$\chi_R$和$\xi_L$都有两个分量,因此我们在诸如$(\chi_R)^\dagger\xi_L$的项里面有四个自由度,而在$\epsilon(\xi_L)^\dagger\xi_L=\xi^a\xi_a$中,像右手旋量一样变换的东西并不依赖于$\xi_L$,我们从而只有两个自由度。关于Majorana旋量实际上还有很多可以说的,它到现在还在进行(实验上的)研究来确定中微子到底属于哪一种。再多说一句,Majorana旋量是``实''的Dirac旋量。这里我打引号的原因是通常实的就是说$\Psi^\star\overset{!}{=}\Psi$。对于旋量来说,这个条件并不Lorentz协变(因为Lorentz变换的表示是复的)。如果我们强制要求我们的标准要求($\Psi^\star\overset{!}{=}\Psi$)的话,对于换一个参考系这事情一般来说就不对了。然而,我们还是给一个对Dirac旋量的Lorentz协变的``实''条件:$\left(\begin{matrix}0&\epsilon\\-\epsilon&0\end{matrix}\right)\Psi^\star\overset{!}{=}\Psi$,二者通常诠释为电荷共轭。这种诠释在第\ref{sec7.1.5}中会进行解释。因此,Majorana旋量实际上描述的是一种粒子,它的电荷共轭粒子,通常叫反粒子,就是它自己。Majorana粒子是自己的反粒子,而Majorana旋量就是Dirac旋量再加一个附加条件:$\Psi_M\equiv\left(\begin{matrix}\xi_L\\\epsilon\xi_L^\star\end{matrix}\right)$或$\Psi_M\equiv\left(\begin{matrix}-\epsilon\chi_R^\star\\\chi_R\end{matrix}\right)$}:
\begin{align}
\label{equ6.7}
I_1:=\chi_{\dot{a}}^T\xi^{\dot{a}}=(\chi_a^\star)^T\xi^{\dot{a}}=(\chi_a)^\dagger\xi^{\dot{a}}=(\chi_L)^\dagger\xi_R
\end{align}
和
\begin{align}
\label{equ6.8}
I_2:=(\xi^a)^T\chi_a=((\xi^a)^\star)^T\chi_a=(\xi_R)^\dagger\chi_L
\end{align}
\ref{sec3.7.7}节讲过,Lorentz不变量总是要把带点的下标和带点的上标,或者不带点的下标和不带点的上标结合。这里再次看到右手旋量和左手旋量需要成对出现。

更进一步的,我们可以构造两种组合,其涉及到一阶导数且满足Lorentz协变,我们一会儿就会看到。但是首先需要理解旋量的导数意味着什么。\ref{sec3.7.8}节讲过如何从旋量构造4-矢量
\[v_{a\dot{b}}=v_\nu\sigma^\nu_{a\dot{b}} \]
其中$v_\nu$像4-矢量一样变换。因此,旋量表述下的导数算符就是
\begin{align}
\label{equ6.9}
\partial_{a\dot{b}}=\partial_\nu\sigma^\nu_{a\dot{b}}
\end{align}

惯例上会定义$\bar\sigma^0=I_{2\times2},\bar\sigma^i=-\sigma^i$,因此我们可以构建Lorentz不变的项
\begin{align}
\label{equ6.10}
I_3:=(\chi_{\dot{a}})^T\partial_\mu(\sigma^\mu)^{\dot{a}b}\chi_b=(\chi_L)^\dagger\partial_\mu\bar\sigma^\mu\chi_L
\end{align}
和
\begin{align}
\label{equ6.11}
I_4:=(\xi^a)^T\partial_\mu(\sigma^\mu)_{a\dot{b}}\xi^{\dot{b}}=(\xi_R)^\dagger\partial_\mu\sigma^\mu\xi_R
\end{align}

除了$(\sigma^\mu)_{a\dot{b}}$之外,我们还需要$(\sigma^\mu)^{a \dot{b}}$。\footnote{原文为$(\sigma^\mu)^{\dot{a} b}$,译文已按勘误表改正。}第一个指标必须是不带点的,第二个指标必须是带点的,这样才可以和其它旋量指标合适的组合。我们通过作用旋量度规两次得到$(\sigma^\mu)^{a\dot{b}}$:
\begin{align}
\label{equ6.12}
\begin{split}
(\sigma^\mu)^{a\dot{b}}&=((\sigma^\mu)^T)^{\dot{b}a}=\epsilon^{\dot{b}\dot{c}}((\sigma^\mu)^T)_{\dot{c}d}(\epsilon^{ad})^T\\
&=\left(\begin{matrix}0&1\\-1&0\end{matrix}\right)(\sigma^\mu)^T\left(\begin{matrix}0&1\\-1&0\end{matrix}\right)^T\\
&=\left(\begin{matrix}0&1\\-1&0\end{matrix}\right)(\sigma^\mu)^T\left(\begin{matrix}0&-1\\1&0\end{matrix}\right)=\bar\sigma_\mu
\end{split}\end{align}
比如,对$\sigma_3$来说,
\[\left(\begin{matrix}0&1\\-1&0\end{matrix}\right)\underbrace{\left(\begin{matrix}1&0\\0&-1\end{matrix}\right)}_{=\sigma_3^T}\left(\begin{matrix}0&-1\\1&0\end{matrix}\right)=\underbrace{\left(\begin{matrix}-1&0\\0&1\end{matrix}\right)}_{\mathclap{=\bar\sigma_3=-\sigma_3}} \]
千万别因为$\partial_\mu$只作用在一个旋量上就困惑,因为我们要在拉格朗日量中使用这些不变量,它们总要进行积分\mpar{我们得到运动方程是从作用量来的,作用量就是拉格朗日量的积分。},因此我们可以进行分部积分得到其它的可能的形式。因此这里的选择是不加约束的\mpar{我们得到对应的运动方程的时候会看的更清楚。我们不论在哪里放$\partial_\mu$我们都会得到相同的运动方程,当然我们也可以把两中可能的组合都放到拉格朗日量里面去。这就会使得式子更长,但是并没有什么新的东西。}。

如果我们定义矩阵\mpar{我们得到有下标的矩阵通过使用矢量指标的升降$\gamma_\mu=\eta_{\mu\nu}\gamma^\nu=\eta_{\mu\nu}\left(\begin{matrix}0&\sigma^\nu\\\bar\sigma^\nu&0 \end{matrix}\right)=\left(\begin{matrix}0&\eta_{\mu\nu}\sigma^\nu\\\eta_{\mu\nu}\bar\sigma^\nu&0 \end{matrix}\right)=\left(\begin{matrix}0&\sigma_\nu\\\bar\sigma_\nu&0 \end{matrix}\right)$,因为$\eta_{\mu\nu}\sigma^\nu=\left(\begin{matrix}1&0&0&0\\0&-1&0&0\\0&0&-1&0\\0&0&0&-1\end{matrix}\right)\left(\begin{matrix}\sigma^0\\\sigma^1\\\sigma^2\\\sigma^3\end{matrix}\right) = \left(\begin{matrix}\sigma^0\\-\sigma^1\\-\sigma^2\\-\sigma^3\end{matrix}\right) = \bar\sigma_\mu$}
\begin{align}
\label{equ6.13}
\gamma^\mu=\left(\begin{matrix}0&\sigma_\mu\\\bar\sigma_\mu&0\end{matrix}\right)\to\gamma_\mu=\left(\begin{matrix}0&\bar\sigma_\mu\\\sigma_\mu&0\end{matrix}\right)
\end{align}
我们可以用Dirac旋量的形式来写我们刚刚发现的不变量
\begin{align}
\label{equ6.14}
\Psi^\dagger\gamma_0\Psi \text{\ \ 和\ \ }\Psi^\dagger\gamma_0\gamma^\mu\partial_\mu\Psi,
\end{align}
因为
\[\Psi^\dagger \gamma_0 \Psi = \left(
    \begin{matrix}
        (\chi_L)^\dagger&(\xi_R)^\dagger
    \end{matrix}
\right) \left(
    \begin{matrix}
    0 & \bar\sigma_0\\
    \sigma_0 & 0
    \end{matrix}
\right) \left(
    \begin{matrix}
    \chi_L\\
    \xi_R
    \end{matrix}
\right) = \underbrace{(\chi_L)^\dagger \bar\sigma_0\xi_R}_{=I_1}+\underbrace{(\xi_R)^\dagger\sigma_0\chi_L}_{=I_2} \]
这与我们之前找到的两个不变量完全一致\mpar{$\sigma_0$只是单位矩阵而已,因此$\bar\sigma_0=\sigma_0$。},并且
\[\begin{split}
\Psi^\dagger\gamma_0\gamma^\mu\partial_\mu\Psi &= \left(\begin{matrix}(\xi_L)^\dagger&(\chi_R)^\dagger\end{matrix}\right)\left(\begin{matrix}0&\bar\sigma_0\\\sigma_0&0\end{matrix}\right)\left(\begin{matrix}0&\sigma^\mu\partial_\mu\\\sigma^\mu\bar\partial_\mu&0\end{matrix}\right)\left(\begin{matrix}\xi_L\\\chi_R\end{matrix}\right)\\
& =\underbrace{(\xi_L)^\dagger\bar\sigma_0\bar\sigma^\mu\partial_\mu\chi_R}_{=I_3}+\underbrace{(\chi_R)^\dagger\sigma_0\sigma^\mu\partial_\mu\xi_L}_{=I_4}
\end{split}\]
就想说过的那样给出了另外两个不变量\mpar{注意$\sigma^\mu\partial_\mu=\partial_\mu\sigma^\mu$,因为$\sigma^\mu$仅仅是常数矩阵而已。}。通常引入记号\footnote{译者注:在一些其它教科书里,一般也会写成$\cancel{\Psi}$。}
\begin{align}
\label{equ6.15}
\bar\Psi:=(\Psi)^\dagger\gamma_0
\end{align}

现在可以写出利用Dirac旋量构造的Lorentz不变的拉格朗日量的完整形式了,它符合\ref{sec4.2}节的所有限制\footnote{译者注:这句话太绕口了。直接看英文会好点:Now we have everything we need to construct a Lorentz-invariant Lagrangian that is in agreement with the restrictions discussed in Sec. \ref{sec4.2} using Dirac spinors}:
\[\mathscr{L}=A\Psi^\dagger\gamma_0\Psi+B\Psi^\dagger\gamma_0\gamma^\mu\partial_\mu\Psi=A\bar\Psi\Psi+ B\bar\Psi\gamma^\mu\partial_\mu\Psi\]
带入常数($A=-m,B=i$),我们得到了完整的{\bf Dirac拉格朗日量}
\begin{align}
\label{equ6.16}
\mathscr{L}_{\text{Dirac}}=-m\bar\Psi\Psi+ i\bar\Psi\gamma^\mu\partial_\mu\Psi=\bar\Psi(i\gamma^\mu\partial_\mu-m)\Psi
\end{align}
注意$\mathscr{L}_{\text{Dirac}}$里面实际上有两个不同的场,因为$\Psi$是复的\mpar{因此左手旋量和右手旋量也都是复的。}。不然就没法Lorentz不变了。更准确的说,我们有
\[\Psi=\Psi_1+i\Psi_2 \]
其中$\Psi_1,\Psi_2$是实数场。通常我们使用两个不同的复数场$\Psi$和$\bar\Psi$比两个实数场要更方便。
把这个拉格朗日量代入到欧拉-拉格朗日方程(方便起见,这里再写一遍),
\begin{align}
\label{equ6.17}
\frac{\partial\mathscr{L}}{\partial\Psi}-\partial_\mu\left(\frac{\partial\mathscr{L}}{\partial(\partial_\mu\Psi)}\right)=0
\end{align}
我们有
\[-m\bar\Psi-i\partial_\mu\bar\Psi\gamma^\mu=0\to(i\partial^\mu\bar\Psi\gamma_\mu+m\bar\Psi)=0 \]
同样将场$\bar\Psi$带入欧拉-拉格朗日方程
\[\frac{\partial\mathscr{L}}{\partial\bar\Psi}-\partial_\mu\left(\frac{\partial\mathscr{L}}{\partial(\partial_\mu\bar\Psi)}\right)=0 \]
我们有
\begin{align}
\label{equ6.18}
(i\gamma_\mu\partial^\mu-m)\Psi=0
\end{align}
这就是著名的{\bf Dirac方程}。注意我们对拉格朗日量进行分部积分也会得到完全一样的东西
\[-m\bar\Psi\Psi+i\bar\Psi\gamma_\mu\partial^\mu\Psi\underbrace{=}_{\mathclap{\text{在作用量积分中分部积分}}}-m\bar\Psi\Psi-(i\partial^\mu\bar\Psi)\gamma_\mu\Psi \]
然后利用欧拉-拉格朗日方程,
\[\to-m\Psi+i\partial_\mu\gamma^\mu\Psi=0 \]
因此拉格朗日量的形式是没有约束的,除了它的反对称性的要求\mpar{你完全可以使用更长的拉格朗日量来包含更多的可能的项,但事实上结果没有区别。}。



\section[Proca方程]{Proca Equation \quad Proca 方程}\label{sec6.4}

现在想找那些根据Lorentz群的$(\frac{1}{2},\frac{1}{2})$表示变换的对象的运动方程。前面已经看到这个表示{\bf 是}矢量表示,因此这件事就比较简单了。考虑一个任意的矢量场$A_\mu$,然后构造所有符合第\ref{sec4.2}节约束的Lorentz不变的项。我们必须结合下标和上标,因为我们在第\ref{sec2.4}节定义的Minkowski空间的标量积是这样的,而标量才是拉格朗日量里面需要的。可能的不变量有\mpar{由于可以分部积分,我们再一次扔掉了$\partial_\mu\partial^\mu A^\nu A_\nu$这项,而仅留着$\partial^\mu A^\nu\partial_\mu A_\nu$。}
\[\begin{split}
I_1=\partial^\mu A^\nu\partial_\mu A_\nu\quad &,\quad I_2=\partial^\mu A^\nu\partial_\nu A_\mu\\
I_3=A^\mu A_\mu\quad &,\quad I_4=\partial^\mu A_\mu
\end{split}\]
而拉格朗日量就写为
\begin{align}\label{equ6.19}
\mathscr{L}_{\text{Proka}} = C_1\partial^\mu A^\nu\partial_\mu A_\nu+C_2\partial^\mu A^\nu\partial_\nu A_\mu+C_3A^\mu A_\mu+C_4\partial^\mu A_\mu
\end{align}
我们可以忽略最后一项$\partial^\mu A_\mu$,因为根据欧拉-拉格朗日方程,它只在运动方程里面提供常数项$C_4$,并不产生什么影响。因此,一阶的$\partial_\mu$项是平凡的。

利用欧拉-拉格朗日方程分别计算每个场分量的运动方程,
\[\frac{\partial\mathscr{L}}{\partial A_\rho}=\partial_\sigma\left(\frac{\partial\mathscr{L}}{\partial(\partial_\sigma A_\rho)}\right) \]
需要非常仔细的处理这些指标。我们先看一下欧拉-拉格朗日方程的右边,并计算与$C_1$有关的项:
\[\begin{split}
&\phantom{ \underbrace{=}_{\mathclap{\text{乘积法则}}} } \partial_\sigma\left(\frac{\partial}{\partial(\partial_\sigma A_\rho)}(C_1\partial^\mu A^\nu\partial_\mu A_\nu)\right)\\
&\underbrace{=}_{\mathclap{\text{product rule}}}C_1\partial_\sigma\left((\partial_\mu A_\nu)\frac{\partial(\partial^\mu A^\nu)}{\partial(\partial_\sigma A_\rho)}+(\partial^\mu A^\nu)\frac{\partial(\partial_\mu A_\nu)}{\partial(\partial_\sigma A_\rho)}\right)\\
&\underbrace{=}_{\mathclap{\text{用度规降指标}}}C_a\partial_\sigma\left((\partial_\mu A_\nu)g^{\mu\kappa}g^{\nu\lambda}\frac{\partial(\partial_\kappa A_\lambda)}{\partial(\partial_\sigma A_\rho)}+(\partial^\mu A^\nu)\frac{\partial(\partial_\mu A_\nu)}{\partial(\partial_\sigma A_\rho)}\right)\\
&=C_a\partial_\sigma\left((\partial_\mu A_\nu)g^{\mu\kappa}g^{\nu\lambda}\delta^\sigma_\kappa\delta^\rho_\lambda+(\partial^\mu A^\nu)\delta^\sigma_\mu\delta^\rho_\nu\right)\\
&=C_1\partial_\sigma(\partial^\sigma A^\rho+\partial^\sigma A^\rho)=2C_1\partial_\sigma\partial^\sigma A^\rho
\end{split} \]
类似的,我们可以计算出
\[\partial_\sigma\left(\frac{\partial}{\partial(\partial_\sigma A^\rho)}(C_2\partial^\mu A^\nu\partial_\nu A_\mu)\right)=2C_2\partial^\rho(\partial_\sigma A^\sigma) \]
因此,\eqref{equ6.19}的拉格朗日量得到的运动方程为
\[2C_3A^\rho=2C_1\partial_\sigma\partial^\sigma A^\rho+2C_2\partial^\rho(\partial_\sigma A^\sigma) \]
带入惯例的常数得到
\begin{align}
\label{equ6.20}
\to m^2A^\rho=\frac{1}{2}\partial_\sigma(\partial^\sigma A^\rho-\partial^\rho A^\sigma)
\end{align}

这被称为{\bf Proca方程组}\mpar{称为方程组是因为对每一个分量$\rho$都有一个方程},这是有质量自旋为$1$的粒子的波动方程。对于无质量$(m=0)$自旋为$1$的粒子,即光子,方程为
\begin{align}
\label{equ6.21}
\to 0=\frac{1}{2}\partial_\sigma(\partial^\sigma A^\rho-\partial^\rho A^\sigma)
\end{align}

这就是没有电流情况下的{\bf 非齐次Maxwell方程组}。通常我们定义电磁张量
\begin{align}
\label{equ6.22}
F^{\sigma\rho}:=\partial^\sigma A^\rho-\partial^\rho A^\sigma
\end{align}
而非齐次Maxwell方程组则是
\begin{align}
\label{equ6.23}
\partial_\rho F^{\sigma\rho}=0
\end{align}
注意我们可以对无质量的自旋$1$场重新写拉格朗日量
\[\mathscr{L}=\frac{1}{2}(\partial^\mu A^\nu\partial_\mu A_\nu - \partial^\mu A^\nu\partial_\nu A_\mu) \]
为
\begin{align}
\label{equ6.24}
\begin{split}
\mathscr{L}_{\text{Maxwell}}&=\frac{1}{4}F^{\mu\nu}F_{\mu\nu}\\
&=\frac{1}{4}(\partial^\mu A^\nu-\partial^\nu A^\mu)(\partial_\mu A_\nu-\partial_\nu A_\mu)\\
&\underbrace{=}_{\mathclap{\text{对哑指标重新命名}}}\frac{1}{4}(\partial^\mu A^\nu\partial_\mu A_\nu - \partial^\mu A^\nu\partial_\nu A_\mu-\partial^\mu A^\nu\partial_\nu A_\mu+\partial^\mu A^\nu\partial_\mu A_\nu)\\
&=\frac{1}{4}(2\partial^\mu A^\nu\partial_\mu A_\nu - 2\partial^\mu A^\nu\partial_\nu A_\mu)\\
&=\frac{1}{2}(\partial^\mu A^\nu\partial_\mu A_\nu - \partial^\mu A^\nu\partial_\nu A_\mu)\checkmark
\end{split}
\end{align}
$\mathscr{L}_{\text{Maxwell}} = \frac{1}{4} F^{\mu \nu} F_{\mu \nu}$是我们惯用的拉格朗日量的形式。等价的,有质量的自旋$1$的场的拉格朗日量可以写成
\begin{align}
\label{equ6.25}
\mathscr{L}_{\text{Proca}} = \frac{1}{4}F^{\mu\nu}F_{\mu\nu}+m^2A_\mu A^\mu=\frac{1}{2}(\partial^\mu A^\nu\partial_\mu A_\nu - \partial^\mu A^\nu\partial_\nu A_\mu)+m^2A_\mu A^\mu
\end{align}

本章导出了描述自由场/粒子们的运动方程。我们想知道这些方程能预言什么实验现象,但在此之前需要更多的方程才行,因为实验永远是涉及相互作用的。比如只能通过其他粒子,比如光子,来测量电子。因此,下一章将推导描述不同的场/粒子之间相互作用的拉格朗日量。
