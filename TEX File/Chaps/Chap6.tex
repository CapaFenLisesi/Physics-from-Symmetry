%!TEX encoding = UTF-8 Unicode

%----------------------------------------------------------------------------------------
%	CHAPTER 6
%----------------------------------------------------------------------------------------

\chapterimage{chapter_head_1.pdf} % Chapter heading image

\chapter[自由场理论]{Free Theory 自由场理论}\label{chap6}

在这章中,我们建立自由的体系的物理理论,也就是说无相互作用的情况下的对称性给出的场\mpar{尽管我们这里针对的是场,后面会看到这里得到的方程同样可以用来处理粒子}。我们会
\begin{itemize}
\item 利用Lorentz群的$(0,0)$表示给出Klein-Gordon方程
\item 利用Lorentz群的$(\tfrac{1}{2},0)\oplus(0,\tfrac{1}{2})$表示给出Dirac方程
\item 利用Lorentz群的矢量$(\tfrac{1}{2},\tfrac{1}{2})$表示给出Proca方程,在无质量极限下退化到著名的Maxwell方程组
\end{itemize}

\section[Lorentz协变性与不变量]{Lorentz Covariance and Invariance Lorentz协变性与不变量}\label{sec6.1}

在下面的章节中,我们会得到粒子物理的标准模型- 这是我们拥有的最好的物理理论 -中的运动方程。我们期待这些房产在所有的惯性系中看起来都一样,因为如果不这样的话,我们就会对每一个可能的参考系都有各自的方程。狭义相对论指出并没有一个特别的参考系,所以这一点是没有意义的。这个术语被称为{\bf Lorentz协变性}。一个Lorentz协变的物体是说它按照某一种给定的Lorentz群来进行变换。比如说,矢量$A_\mu$,通过$\left(\frac{1}{2},\frac{1}{2}\right)$表示来进行变换,因此是Lorentz协变的;这其实是说,$A_\mu\to A'_{\mu'}$,但实际上这两个指代同一个东西,并不是完全不同的。另一方面,比如说,$A_1+A_3$就不是Lorentz协变的,因为它不按照Lorentz群的某种表示来变换。但这并不是说我们不知道它是如何变换的,因为它的变换规律从$A_\mu$的变换规律里面可以简单地得出,但是在不同的惯性系中它看起来完全不一样。在一个``推动''的参考系中它看起来或许像$A_2+A_4$。只涉及到Lorentz协变的东西的方程被称为Lorentz协变方程,比如
\[A_\mu+7B_\mu+C_\nu A^\nu D_\mu = 0 \]
就是一个Lorentz协变方程,因为在另一个参考系中,它看起来就是
\[A_\mu+7B_\mu+\underbrace{C_\nu A^\nu}_{\mathclap{=\text{一个Lorentz标量,根据$(0,0)$表示来进行变换}}} D_\mu = A'_\mu+7B'_\mu+C_\nu A^\nu D'_\mu = 0 \]